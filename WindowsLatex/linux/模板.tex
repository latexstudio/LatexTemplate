%%%%%%%%%%%%%%%%%%%%%%%%%%%%%%%%%%%%%%%%%%%%%%%%%%%%%%%%%%%%%%%%%%%%%%%%%%%%%%%%%%%%%%%%%%%%%%%%%%%%%%%%%%
%%                             有中文的记得用XeLaTeX来编译
%%%%%%%%%%%%%%%%%%%%%%%%%%%%%%%%%%%%%%%%%%%%%%%%%%%%%%%%%%%%%%%%%%%%%%%%%%%%%%%%%%%%%%%%%%%%%%%%%%%%%%%%%%
%\documentclass[a4paper,11pt]{ctexart}
\documentclass[a4paper,11pt]{article}
\usepackage{graphicx}% 用于图像
\usepackage{amsmath}
\usepackage{amsbsy,amscd,amsgen,amsmath,amsopn,amsxtra,amsthm,upref}
%%%%%%%%%%%%%%%%%%%%%%%%%以下为中英文字体设置%%%%%%%%%%%%%%%%%%%%%%%%%%%%%%%%%%%%%%%%%%%
\usepackage{times}
\usepackage{fontspec} % XeLaTeX相关字体字库,xunicode,xltxtra
\XeTeXlinebreaklocale "zh"
\XeTeXlinebreakskip = 0pt plus 1pt minus 0.1pt

%%%%%%%%%%%%%%%%%%%%%%%%%%%%%%%%%%%%%%%%%%%%%%%%%
\newfontfamily\kai{AR PL UKai CN}%字体重命名,方便使用
\newfontfamily\ming{AR PL UMing CN}
\newfontfamily\hei{WenQuanYi Zen Hei}
\setmainfont{WenQuanYi Zen Hei}%文档正文默认字体,设置衬线字体
%\setmainfont{Times New Roman} %设置默认英文字体
%%字体字号名字的重定义(方便快捷使用)
\newcommand{\sanhao}{\fontsize{16pt}{24pt}\selectfont}      % 三号, 1.5倍行距
\newcommand{\sihao}{\fontsize{14pt}{21pt}\selectfont}       % 四号, 1.5倍行距
\newcommand{\xiaosi}{\fontsize{12pt}{18pt}\selectfont}      % 小四, 1.5倍行距
\newcommand{\wuhao}{\fontsize{10.5pt}{10.5pt}\selectfont}   % 五号, 单倍行距
%\newfontfamily\song{SimSun} %宋体
%\newfontfamily\hei{SimHei} %黑体
%\newfontfamily\roman{Times New Roman}

%%%%%%%%%%%%%%%%%%%%%%%%%以上为中英文字体设置%%%%%%%%%%%%%%%%%%%%%%%%%%%%%%%%%%%%%%%%%%%

%%%%%%%%%%%%%%%%%%%%%%%以下是版面控制部分%%%%%%%%%%%%%%%%%%%%%%%%%%%%%%%%%%%%%%%%%%%%%%
\usepackage{geometry}\geometry{left=2.75cm,right=2.5cm,top=2.5cm,bottom=2.5cm}
\usepackage{indentfirst}             % 首行缩进
\usepackage[perpage,symbol]{footmisc}% 脚注控制
\usepackage[sf]{titlesec}            % 控制标题
\usepackage{titletoc}                % 控制目录
%\titlecontents{section}[0pt]{\addvspace{2pt}\filright}
%              {\contentspush{\thecontentslabel\ }}
%              {}{\titlerule*[8pt]{.}\contentspage}
%                                     % 添加section在目录里的点号
\usepackage{setspace}                % 调节行间距   
\usepackage{booktabs}                % 用于表格中加下划线
\usepackage{fancyhdr}                % 页眉页脚
\usepackage{type1cm}                 % 控制字体大小
\usepackage{indentfirst}             % 首行缩进
\usepackage{makeidx}                 % 建立索引
\usepackage{textcomp}                % 千分号等特殊符号
%\usepackage{layouts}                 % 打印当前页面格式
\usepackage{cite}                    % 支持引用
\setlength{\skip\footins}{0.5cm}     % 脚注与正文的距离
%%%%%%%%%%%%%%%%%%%%%%%%%%以上是版面控制部分%%%%%%%%%%%%%%%%%%%%%%%%%%%%%%%%%%%%%%%%%%%%%






%%%%%%%%%%%%%%%%%%%%%%%%%%%%%以下进入正文%%%%%%%%%%%%%%%%%%%%%%%%%%%%%%%%%%%%%%%%%%%%%%
\title{\sanhao{LATEX模板}}
\author{\xiaosi{吴鸿毅}} 
\date{2014年4月30日} %添加时间
%%以上三个信息会在正文中有 \maketitle 的地方出现

\begin{document}
\maketitle
\begin{titlepage}   %封面

\begin{spacing}{1.8}    %行间距增为1.8倍
\begin{center}
\begin{tabular}{rc}
\sanhao 论文题目:& \sanhao 暂时还不知道\\ \cmidrule(r){2-2} %第二格加下划线
\sanhao 学生姓名:& \sanhao 吴 \qquad 鸿 \qquad 毅\\ \cmidrule(r){2-2}
\sanhao 学\ \ \ \quad 号:& \sanhao 2011151416\\ \cmidrule(r){2-2}
\sanhao 专\ \ \ \quad 业: & \sanhao 核~工~程~与~核~技~术 \\ \cmidrule(r){2-2}
\sanhao 指导老师: & \sanhao 无 \quad 名 \quad 氏\\ \cmidrule(r){2-2}
\sanhao 学\ \ \ \quad 院: & \sanhao 核~科~学~与~技~术~学~院\\ \cmidrule(r){2-2}
\end{tabular}
\end{center}
\end{spacing}
\centerline{\normalsize 哈尔滨工程大学}
\centerline{\sanhao 2014年4月30日}
\end{titlepage}
\thispagestyle{empty}% 首页无页眉页脚
\newpage%另起一页

\begin{spacing}{2}
\tableofcontents                    % 插入目录,需要编译两次才能出现
\end{spacing}
\thispagestyle{empty}               % 首页无页眉页脚

%%%%%%%%%%%%%%%%%%%%%%%%%%%%%以下是中文摘要、关键词%%%%%%%%%%%%%%%%%%%%%%%%%%%%%%%%%%%%
\clearpage %双面打印(openright) 用\cleardoublepage,刷新页面信息,为了添加目录章节后页码不乱
\pagenumbering{arabic} %自此处页码开始计数       
\addcontentsline{toc}{section}{\textbf{\xiaosi{摘要}}} %创建虚拟章节,便于将摘要部分添加到目录






%%%%%%%%%%%%%%%%%%%%%%%%%%%%%以下为论文引言部分%%%%%%%%%%%%%%%%%%%%%%%%%%%%%%%%%%%%%%%%%
\section{\textbf{\xiaosi{\hei 引言}}}

{\xiaosi{\ming{在一般的本科生泛函分析教材中, $\cdots$ 如~$L^p(\mathbf{E})$~和~$l^p$~}}}\\
%%%%%%%%%%%%%%%%%%%%%%%%%%%%%以上为论文引言部分%%%%%%%%%%%%%%%%%%%%%%%%%%%%%%%%%%%%%%%%%%


{\xiaosi{\ming{
%%%%%%%%%%%%%%%%%%%%%%%%%%%%%以下为论文第二部分%%%%%%%%%%%%%%%%%%%%%%%%%%%%%%%%%%%%%%%%%%
\section{\textbf{\xiaosi{\hei 数列空间和函数空间的定义}}}

以下给出六种典型的数列空间和函数空间的定义,文字叙述和符号表示依照文献\cite{cankaowenxiao111}.\\

%%%%%%%%%%%%%%%%%%%%%%%%以上为论文正文第二部分%%%%%%%%%%%%%%%%%%%%%%%%%%%%%%%%%%%%%%%%%%%%%%%


%%%%%%%%%%%%%%%%%%%%%%%%以下为论文正文第三部分%%%%%%%%%%%%%%%%%%%%%%%%%%%%%%%%%%%%%%%%%%%%%%
\section{\textbf{\xiaosi\hei 六类空间各自的性质}}

{\subsection{\textbf{\xiaosi\hei{$l^p$和$L^p(\mathbf{E})$}}}}
$l^p$~和~$L^p(\mathbf{E})$~都可分.
$\cdots$
%%%%%%%%%%%%%%%%%%%%%%%%%%%以上为论文正文第三部分%%%%%%%%%%%%%%%%%%%%%%%%%%%%%%%%%%%%%%%%%%


%%%%%%%%%%%%%%%%%%%%%%%%%%%以下为论文正文第四部分%%%%%%%%%%%%%%%%%%%%%%%%%%%%%%%%%%%%%%%%%%
{\section{\textbf{\xiaosi\hei 六种空间之间的一些联系}}



文献\cite{cankao4}给出了~$1\leq p\leq 2$~时的~$L^p(\mathbf{E})$~上的~Fourier~变换的构造过程,并指出当
~$p>2$~时在广义函数的意义下~$L^p(\mathbf{E})$~仍可导入~Fourier~变换.\ 问题在于~$p\neq 2$~时~Fourier
~变换能否构成~$L^p(\mathbf{E})$~与~$l^p$~之间的保范同构

\begin{equation}
a^x+y \neq a^{x+y}
\end{equation}
$\lambda,\xi,\pi,\mu,\Phi,\Omega$

$a_{1}$ \qquad $x^{2}$ \qquad
$e^{-\alpha t}$ \qquad
$a^{3}_{ij}$\\
$e^{x^2} \neq {e^x}^2$

$\sqrt{x}$ \qquad
$\sqrt{ x^{2}+\sqrt{y} }$
\qquad $\sqrt[3]{2}$\\[3pt]
$\surd[x^2 + y^2]$

$\overline{m+n}$ \qquad
$\underline{m+n}$

\[\underbrace{ a+b+ \cdots +z }_{26}\]

\begin{displaymath}
y=x^{2}\qquad y’=2x\qquad y’’=2
\end{displaymath}

\begin{displaymath}
\vec a\quad\overrightarrow{AB}
\end{displaymath}

\begin{displaymath}
v = {\sigma}_1 \cdot {\sigma}_2
{\tau}_1 \cdot {\tau}_2
\end{displaymath}

%\arccos \cos
%\csc \exp
%\ker
%\limsup \min
%\arcsin \cosh \deg \gcd \lg \ln \Pr
%\arctan \cot \det \hom \lim \log \sec
%\arg \coth \dim \inf \liminf \max \sin
%\sinh \sup \tan \tanh 

\[\lim_{x \rightarrow 0}
\frac{\sin x}{x}=1\]
%\rightarrow
 $X \neq \sqrt{Y}$   $\hat{a}\check{a}\dot{a}\tilde{a}\ddot{a}\acute{a}\breve{a}$
 $111
\alpha
\beta
\gamma
\delta
\epsilon
\varepsilon
\zeta
\eta
\theta
\vartheta
\iota
\kappa
\lambda
\mu
\nu
\xi
\pi
\varpi
\rho
\varrho
\sigma
\varsigma
\tau
\upsilon
\phi
\varphi
\chi
\psi
\omega
111
$
}}}





\end{document}