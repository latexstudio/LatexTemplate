%%% C++技术储备.tex --- 
%% 
%% Description: 
%% Author: Hongyi Wu(吴鸿毅)
%% Email: wuhongyi@qq.com 
%% Created: 六 4月  5 10:16:59 2014 (+0800)
%% Last-Updated: 六 4月  5 10:27:17 2014 (+0800)
%%           By: Hongyi Wu(吴鸿毅)
%%     Update #: 2
%% URL: http://wuhongyi.cn 

\documentclass[11pt,a4paper,titlepage]{article}
%横向排版(可省略参数 landscape),文章的标题单独占据一页(可省略参数 titlepage),标题与文章正文排在同一页面(可省略参数 notitlepage),单双面排版(可省略参数 oneside 与 twoside)

\usepackage[top=1.2in,bottom=1.2in,left=1.2in,right=1in]{geometry}%页面边距设置
\usepackage{indentfirst}% 首行缩进


\usepackage{color,xcolor} % 支持彩色文本、底色、文本框等
\usepackage{listings} % 粘贴源代码
\lstloadlanguages{c++} % 所要粘贴代码的编程语言
\lstset{
language=,tabsize=4, keepspaces=true,
xleftmargin=2em,xrightmargin=2em, aboveskip=1em,
backgroundcolor=\color{lightgray},% 定义背景颜色
frame=none,% 表示不要边框
keywordstyle=\color{blue}\bfseries,
breakindent=22pt,
numbers=left,stepnumber=1,numberstyle=\tiny,
basicstyle=\footnotesize,
showspaces=false,
flexiblecolumns=true,
breaklines=true, breakautoindent=true,breakindent=4em,
escapeinside={/*@}{@*/}
}


\usepackage{fontspec}%使用其提供的 \setmainfont 命令可设定文稿正文中的中文字体
\XeTeXlinebreaklocale "zh"
\XeTeXlinebreakskip = 0pt plus 1pt minus 0.1pt
%系统中文查看命令:fc-list :lang=zh
\newfontfamily\kai{AR PL UKai CN}%字体重命名,方便使用
\newfontfamily\ming{AR PL UMing CN}
\setmainfont{WenQuanYi Zen Hei}%文档正文默认字体,设置衬线字体
%\setsansfont {}%设定无衬线中文字体
%Sans Serif 无衬线字体 楷体、黑体、幼圆 \sansfont 。Serif 衬线字体 对应中文的 宋、仿宋\mainfont
\usepackage{times}% 包括 Times Roman + Helvetica + Courier
\usepackage{palatino} % 包括 Palatino + Helvetica + Courier
\usepackage{newcent} % 包括 New Century Schoolbook + Avant Garde + Courier
\usepackage{bookman} % 包括 Bookman + Avant Garde + Courier

%%%%%%%%%% 数学符号公式 %%%%%%%%%%
\usepackage{latexsym}
\usepackage{amsmath} % AMS LaTeX宏包
\usepackage{amssymb} % 用来排版漂亮的数学公式
\usepackage{amsbsy}
\usepackage{amsthm}
\usepackage{amsfonts}
\usepackage{mathrsfs} % 英文花体字体
\usepackage{bm} % 数学公式中的黑斜体
\usepackage{relsize} % 调整公式字体大小:\mathsmaller,\mathlarger 
\usepackage{caption2} % 浮动图形和表格标题样式



\usepackage[pagestyles]{titlesec}%章节标题位置center,页眉页脚设置
\newpagestyle{main}{
\sethead{\small\S\,\thesection\quad\sectiontitle}{}{$\cdot$~\thepage~$\cdot$}%页眉
\setfoot{\kai 哈尔滨工程大学}{\kai 核科学与技术学院}{\kai 吴鸿毅}%页脚
\footrule%画页脚线
\headrule%画页眉线
}
\pagestyle{main}


\title{\Huge\kai C++技术储备}
\author{\ming 吴鸿毅\\
            wuhongyi@qq.com}
\date{\ming 2014年4月29日}

\usepackage{hyperref} %保证它是文档导言区的最后一行命令,超链接

\begin{document}
\maketitle%显示标题信息
\tableofcontents%插入目录,需要编译两次才能出现
\newpage%另起一页

%%%%%%%%%%%%%%%%%%%%%%%%%%%%%%%%%%%%%%%%%%%%%%%%%%%%%%%%%%%%
\section{\kai map}
\subsection{\kai map的功能}
\ming map是一类关联式容器。它的特点是增加和删除节点对迭代器的影响很小,除了那个操作节点,对其他的节点都没有什么影响。对于迭代器来说,可以修改实值,而不能修改key。
自动建立Key - value的对应。key 和 value可以是任意你需要的类型。
根据key值快速查找记录,查找的复杂度基本是Log(N),如果有1000个记录,最多查找10次,1,000,000个记录,最多查找20次。\\
快速插入Key - Value 记录。\\
快速删除记录\\
根据Key 修改value记录。\\
遍历所有记录。\\
\subsection{\kai map}
\subsubsection{\kai map根据second进行排序}
\begin{lstlisting}[language=C++, numbers=left]
#include<map>
#include <vector>
#include <algorithm>
#include<iostream>
using namespace std;

typedef pair<int, double> PAIR;

struct CmpByValue {
  bool operator()(const PAIR& lhs, const PAIR& rhs) {
    return lhs.second < rhs.second;
  }
};

//输出重载
ostream& operator<<(ostream& out, const PAIR& p) {
  return out << p.first << "\t" << p.second;
}

int main(int argc, char *argv[])
{
  map<int, double> aaa;
  aaa[0] = 90.1;
  aaa[1] = 79.9;
  aaa[2] = 92.69;
  aaa.insert(make_pair(3,99.12));
  aaa.insert(make_pair(4,86.09));
  //把map中元素转存到vector中 
  vector<PAIR> bbb(aaa.begin(), aaa.end());
  sort(bbb.begin(), bbb.end(), CmpByValue());
  //输出
  for (int i = 0; i != bbb.size(); ++i) {
    cout << bbb[i]<< endl;
  }
  return 0;
}
\end{lstlisting}

\subsubsection{\kai map查找first所对应的second}
\begin{lstlisting}[language=C++, numbers=left]
#include <map>
#include <cstring>
#include <iostream>

  map<string,double> radius;
  map <string,double>::iterator cp;
  
  radius.insert(pair<string,double> (".OH+.OH",0.1416));
  radius.insert(pair<string,double> (".OH+eaq-",0.4525));
  radius.insert(pair<string,double> (".OH+H.",0.2697));
  radius.insert(pair<string,double> (".OH+H2",0.00076));
  radius.insert(pair<string,double> (".OH+H2O2",0.00061));
  radius.insert(pair<string,double> ("eaq-+eaq-",0.0807));
  radius.insert(pair<string,double> ("eaq-+H.",0.2873));
  radius.insert(pair<string,double> ("eaq-+H3O+",0.1664));
  radius.insert(pair<string,double> ("eaq-+O",0.3804));
  radius.insert(pair<string,double> ("H.+H.",0.0944));
  radius.insert(pair<string,double> ("H.+H2O2",0.00144));
  radius.insert(pair<string,double> ("H.+O",0.2904));

  cp=radius.find(".OH+H.");
  cout<<cp->second<<endl;
\end{lstlisting}

\subsubsection{\kai 在map中插入元素}
改变map中的条目非常简单,因为map类已经对[]操作符进行了重载\\
enumMap[1] = "One";\\
enumMap[2] = "Two";\\
.....\\

这样非常直观,但存在一个性能的问题。插入2时,先在enumMap中查找主键为2的项,没发现,然后将一个新的对象插入enumMap,键是2,值是一个空字符串,插入完成后,将字符串赋为"Two"; 该方法会将每个值都赋为缺省值,然后再赋为显示的值,如果元素是类对象,则开销比较大。我们可以用以下方法来避免开销:\\
enumMap.insert(map<int, CString> :: value\_type(2, "Two"))\\

\subsubsection{\kai 查找并获取map中的元素 }
下标操作符给出了获得一个值的最简单方法:
CString tmp = enumMap[2];

但是,只有当map中有这个键的实例时才对,否则会自动插入一个实例,值为初始化值。

我们可以使用Find()和Count()方法来发现一个键是否存在。

查找map中是否包含某个关键字条目用find()方法,传入的参数是要查找的key,在这里需要提到的是begin()和end()两个成员,分别代表map对象中第一个条目和最后一个条目,这两个数据的类型是iterator.

int nFindKey = 2; //要查找的Key

//定义一个条目变量(实际是指针)
\begin{lstlisting}[language=C++, numbers=left]
UDT_MAP_INT_CSTRING::iterator it= enumMap.find(nFindKey);
if(it == enumMap.end()) {
//没找到
}
else {
//找到
}
\end{lstlisting}
通过map对象的方法获取的iterator数据类型是一个std::pair对象,包括两个数据 iterator->first 和 iterator->second 分别代表关键字和存储的数据

\subsubsection{\kai map的插入,查找,遍历及删除的例子}
\begin{lstlisting}[language=C++, numbers=left]
#include <map>
#include <string>
#include <iostream>
using namespace std;

void map_insert(map < string, string > *mapStudent, string index, string x)
{
mapStudent->insert(map < string, string >::value_type(index, x));
}

int main(int argc, char **argv)
{
char tmp[32] = "";
map < string, string > mapS;

//insert element
map_insert(&mapS, "192.168.0.128", "xiong");
map_insert(&mapS, "192.168.200.3", "feng");
map_insert(&mapS, "192.168.200.33", "xiongfeng");

map < string, string >::iterator iter;

cout << "We Have Third Element:" << endl;
cout << "-----------------------------" << endl;

//find element
iter = mapS.find("192.168.0.33");
if (iter != mapS.end()) {
cout << "find the elememt" << endl;
cout << "It is:" << iter->second << endl;
} else {
cout << "not find the element" << endl;
}

//see element
for (iter = mapS.begin(); iter != mapS.end(); iter ) {

cout << "| " << iter->first << " | " << iter->
second << " |" << endl;

}
cout << "-----------------------------" << endl;

map_insert(&mapS, "192.168.30.23", "xf");

cout << "After We Insert One Element:" << endl;
cout << "-----------------------------" << endl;
for (iter = mapS.begin(); iter != mapS.end(); iter ) {

cout << "| " << iter->first << " | " << iter->
second << " |" << endl;
}

cout << "-----------------------------" << endl;

//delete element
iter = mapS.find("192.168.200.33");
if (iter != mapS.end()) {
cout << "find the element:" << iter->first << endl;
cout << "delete element:" << iter->first << endl;
cout << "=================================" << endl;
mapS.erase(iter);
} else {
cout << "not find the element" << endl;
}
for (iter = mapS.begin(); iter != mapS.end(); iter ) {

cout << "| " << iter->first << " | " << iter->
second << " |" << endl;

}
cout << "=================================" << endl;

return 0;
}
\end{lstlisting}

%%%%%%%%%%%%%%%%%%%%%%%%%%%%%%%%%%%%%%%%%%%%%%%%%%%%%%%%%%%%
\section{\kai string}
\subsection{\kai string}
\subsubsection{\kai string对象和C字符串之间的转换}
\ming C++能够执行自动类型转换,所以可以将C字符串储存在string类型的变量中。例如以下代码可以很好地工作:
\begin{lstlisting}[language=C++, numbers=left]
char a_c_string[]=''This is my C string.'';
string string_variable;
string_nariable=a_c_string;
\end{lstlisting}
{~~~~}string对象不能自动转换为C字符串。为了获得对应于string对象的一个C字符串,必须执行显示的类型转换。这需要用到string成员函数c\_str()。用strcpy来复制字符串。
\begin{lstlisting}[language=C++, numbers=left]
strcpy(a_c_string,string_variable.c_str());
\end{lstlisting}
注意,需要用strcpy函数来进行复制。成员函数c\_str()返回与string调用对象对应的一个C字符串。

\subsection{\kai sstream}
\subsubsection{\kai string型转为double型}
\begin{lstlisting}[language=C++, numbers=left]
#include<vector>
#include<cstring>
#include<sstream>
#include <iostream>
using namespace std;

int main( ) {
    string real = "12.32 12 35 25.3 36.366";
    stringstream ss( real );
    vector< double > vd;

    // Collect all real numbers.
    double temp;
    while( ss >> temp )
      { vd.push_back( temp );
    	cout<<temp<<endl;}
    // Create the array.
    double *dbl_array = new double[ vd.size( ) ];
    for( int i = 0; i < vd.size( ); ++i )
        dbl_array[ i ] = vd[ i ];
  string a="$12.m";
  cout<<a[0]<<" "<<a[1]<< " "<<a[2]<<" "<<a[3]<<endl;
}
\end{lstlisting}












\end{document}
%%%%%%%%%%%%%%%%%%%%%%%%%%%%%%%%%%%%%%%%%%%%%%%%%%%%%%%%%%%%%%%%%%%%%%
%%% C++技术储备.tex ends here
