%%% C++.tex --- 
%% 
%% Description: 
%% Author: Hongyi Wu(吴鸿毅)
%% Email: wuhongyi@qq.com 
%% Created: 六 4月  5 10:16:59 2014 (+0800)
%% Last-Updated: 六 4月  5 10:27:17 2014 (+0800)
%%           By: Hongyi Wu(吴鸿毅)
%%     Update #: 2
%% URL: http://wuhongyi.cn 

\documentclass[11pt,a4paper,titlepage]{book}

\usepackage{fontspec}%使用其提供的 \setmainfont 命令可设定文稿正文中的中文字体
\usepackage[top=1.2in,bottom=1.2in,left=1.2in,right=1in]{geometry}%页面边距设置
\XeTeXlinebreaklocale "zh"
\XeTeXlinebreakskip = 0pt plus 1pt minus 0.1pt

%系统中文查看命令:fc-list :lang=zh
\newfontfamily\kai{AR PL UKai CN}%字体重命名,方便使用
\setmainfont{WenQuanYi Zen Hei}%文档正文默认字体,设置衬线字体
%\setsansfont {}%设定无衬线中文字体
%Sans Serif 无衬线字体 楷体、黑体、幼圆 \sansfont 。Serif 衬线字体 对应中文的 宋、仿宋\mainfont

\usepackage[pagestyles]{titlesec}%章节标题位置center,页眉页脚设置
\newpagestyle{main}{
\sethead{\small\S\,\thesection\quad\sectiontitle}{}{$\cdot$~\thepage~$\cdot$}%页眉
\setfoot{哈尔滨工程大学}{核科学与技术学院}{\kai 吴鸿毅}%页脚
\footrule%画页脚线
\headrule%画页眉线
}
\pagestyle{main}

\usepackage{supertabular}%长表格

\usepackage{listings}
\usepackage{color}



\title{C++库使用}
\author{\kai 吴鸿毅 \and wuhongyi@qq.com}
\date{2014年4月5日}

\usepackage{hyperref} %保证它是文档导言区的最后一行命令,超链接

\begin{document}
\maketitle%显示标题信息
\tableofcontents%插入目录,需要编译两次才能出现
\newpage%另起一页

\chapter{C library}
%===========================================================
\section{cmath}
%---------------------------------------------------------
\begin{supertabular}{cl}
\hline
函数& 功能\\
\hline
abs& 绝对值\\
acos& 反余弦\\
acosh& 反双曲余弦值(C++11)\\
asin& 反正弦\\
asinh& 反双曲正弦值(C++11)\\
atan& 反正切\\
atan2& y/x的反正切\\
atanh& 反双曲正切值(C++11)\\
cbrt& 立方根(C++11)\\
ceil& 上取整,不小于给定值的最近整数\\
copysign& 以第二个参数y的符号(正或负)返回第一个参数x\\
cos& 余弦\\
cosh& 双曲余弦值\\
erf& 误差函数(也称之为高斯误差函数)(C++11)\\
erfc& 余补误差函数1-erf() (C++11)\\
exp& 以自然常数e为底的指数函数\\
exp2& 以2为底的指数函数(C++11)\\
expm1& 函数返回 exp(x) - 1(C++11)\\
fabs& 返回值为浮点型的绝对值\\
fdim& 如果X>Y返回X-Y,否则返回0(C++11)\\
floor& 下取整,不大于给定值的最近整数\\
fma& 返回x*y+z,整个运算完成之后,只进行一次舍入(C++11)\\
fmax& 返回参数较大值(C++11)\\
fmin& 返回参数较小值(C++11)\\
fmod& 求余,获得浮点数除法操作的余数\\
fpclassify& (C++11)\\
frexp& 把一个浮点数分解为尾数和指数,其中 x = 尾数 * 2\^{}指数,尾数为[0.5,1) \\
hypot& 计算两个数平方的和的平方根。对于给定的直角三角形的两个直角边,求其斜边的长度(C++11)\\
ilogb& 把一个浮点数分解为尾数和指数,返回指数值。其中 x = 尾数 * 2\^{}指数,尾数为[1,2)(C++11)\\
isfinite& 返回x是否是一个有限值。一个有限值是任何浮点值,既不是无限的也不是非数字(C++11)\\
isgreater& 返回是否x大于y(C++11)\\
isgreaterequal& 返回是否x大于或等于y(C++11)\\
isinf& 返回是否x是一个无穷大的值(正无穷大或负无穷大)。(C++11)\\
isless& 返回是否x小于y(C++11)\\
islessequal& 返回是否x小于或等于y((C++11)\\
islessgreater& 返回是否x小于或大于y(C++11)\\
isnan& 返回是否x是非数字(C++11)\\
isnormal& 返回是否x是一个正常值,即不是无穷的、非数字、零、低于正常的(C++11)\\
isunordered& 返回是否x或y是非数字。检测两个浮点数是否是无序的,只要有一个是 NaN,就不能比较(C++11)\\
ldexp& 计算value乘以2的exp次幂 ( value * ( 2\^{}exp ) ),其中exp为整型\\
lgamma& 返回X的伽玛函数的绝对值的自然对数(C++11)\\
llrint& 返回最靠近x的整数???(C++11)\\
llround& 返回最靠近x的整数???(C++11)\\
log& 计算自然(以 e 为底)对数(ln(x))\\
log10& 计算普通(以 10 为底)对数(log10(x))\\
log1p& 计算 1 与给定值 x 的和(1+x)的自然对数(ln(1+x))(C++11)\\
log2& 计算给定数的以 2 为底的对数(log2(x))(C++11)\\
logb& 计算给定数的以 2 为底的对数,x取绝对值(log2(|x|))(C++11)\\
lrint& 返回最靠近x的整数???(C++11)\\
lround& 返回最靠近x的整数???(C++11)\\
modf& 将一个浮点数分解为整数及小数部分\\
nan& 将执行时定义的字符串作为静态化非数型(Quiet NaN)操作所需的值(C++11)\\
nanf& 将执行时定义的字符串作为静态化非数型(Quiet NaN)操作所需的值(C++11)\\
nanl& 将执行时定义的字符串作为静态化非数型(Quiet NaN)操作所需的值(C++11)\\
nearbyint& (C++11)\\
nextafter& (C++11)\\
nexttoward& (C++11)\\
pow& 幂运算\\
remainder& 获得浮点数除法操作的带符号余数。取到最近的,最小的余数(C++11)\\
remquo& 获得浮点数除法操作的带符号余数,且返回符号及操作结果的最后三位组成的整数(C++11)\\
rint& 返回最靠近x的整数???(C++11)\\
round& 返回最靠近x的整数???(C++11)\\
scalbln& 计算scalbln(x,n) = x * 2\^{}n,其中n为long int 型(C++11)\\
scalbn& 计算scalbn(x,n) = x * 2\^{}n,其中n为 int 型(C++11)\\
signbit& 返回是否x的符号是负的,可以应用到无穷、非数字、零;如果零无符号,它是正的(C++11)\\
sin& 正弦函数\\
sinh& 双曲正弦\\
sqrt& 计算平方根\\
tan& 正切函数\\
tanh& 双曲正切\\
tgamma& 返回X的伽玛函数(C++11)\\
trunc& 幅度(到 0 的距离,即绝对值)不大于给定值的最近整数(C++11)\\
\hline
\end{supertabular}

\begin{lstlisting}[language=C++, numbers=left]
#include <stdio.h>
#include <string.h>
int main ()
{
  char str[] = "This is a sample string";
  char * pch;
  printf ("Looking for the 's' character in \"%s\"...\n",str);
  pch=strchr(str,'s');
  while (pch!=NULL)
  {
    printf ("found at %d\n",pch-str+1);
    pch=strchr(pch+1,'s');
  }
  return 0;
}
\end{lstlisting}


\chapter{Containers}
%===========================================================


\chapter{Input/Output}
%===========================================================



\chapter{Multi-threading}
%===========================================================



\chapter{Other}
%===========================================================




\end{document}
%\chapter{测试}
%\section{\kai 大标题}
%\subsection{\kai 中标题}

%%%%%%%%%%%%%%%%%%%%%%%%%%%%%%%%%%%%%%%%%%%%%%%%%%%%%%%%%%%%%%%%%%%%%%
%%% C++.tex ends here
