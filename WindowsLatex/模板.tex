%%%%%%%%%%%%%%%%%%%%%%%%%%%%%%%%%%%%%%%%%%%%%%%%%%%%%%%%%%%%%%%%%%%%%%%%%%%%%%%%%%%%%%%%%%%%%%%%%%%%%%%%%%
%%                             有中文的记得用XeLaTeX来编译
%%%%%%%%%%%%%%%%%%%%%%%%%%%%%%%%%%%%%%%%%%%%%%%%%%%%%%%%%%%%%%%%%%%%%%%%%%%%%%%%%%%%%%%%%%%%%%%%%%%%%%%%%%
\documentclass[a4paper,11pt]{article}
\usepackage{graphicx}% 用于图像


%%%%%%%%%%%%%%%%%%%%%%%%%以下为中英文字体设置%%%%%%%%%%%%%%%%%%%%%%%%%%%%%%%%%%%%%%%%%%%
\usepackage{times}
\usepackage{fontspec,xunicode,xltxtra} % XeLaTeX相关字体字库
\XeTeXlinebreaklocale "zh"
\XeTeXlinebreakskip = 0pt plus 1pt minus 0.1pt
%%%%%%%%%%%%%%%%%%%%%%%%%%%%%%%%%%%%%%%%%%%%%%%%%
%%字体字号名字的重定义(方便快捷使用)
\newcommand{\sanhao}{\fontsize{16pt}{24pt}\selectfont}      % 三号, 1.5倍行距
\newcommand{\sihao}{\fontsize{14pt}{21pt}\selectfont}       % 四号, 1.5倍行距
\newcommand{\xiaosi}{\fontsize{12pt}{18pt}\selectfont}      % 小四, 1.5倍行距
\newcommand{\wuhao}{\fontsize{10.5pt}{10.5pt}\selectfont}   % 五号, 单倍行距
\newfontfamily\song{SimSun} %宋体,这个应该是系统默认有的
\newfontfamily\hei{SimHei} %黑体,这个应该是系统默认有的
\newfontfamily\kai{simkai.ttf}%楷体,加字体的方法
\newfontfamily\roman{Times New Roman}

%%%%%%%%%%%%%%%%%%%%%%%%%以上为中英文字体设置%%%%%%%%%%%%%%%%%%%%%%%%%%%%%%%%%%%%%%%%%%%

%%%%%%%%%%%%%%%%%%%%%%%以下是版面控制部分%%%%%%%%%%%%%%%%%%%%%%%%%%%%%%%%%%%%%%%%%%%%%%
\usepackage{geometry}\geometry{left=2.75cm,right=2.5cm,top=2.5cm,bottom=2.5cm}
\usepackage{indentfirst}             % 首行缩进
\usepackage[perpage,symbol]{footmisc}% 脚注控制
\usepackage[sf]{titlesec}            % 控制标题
\usepackage{titletoc}                % 控制目录
\titlecontents{section}[0pt]{\addvspace{2pt}\filright}
              {\contentspush{\thecontentslabel\ }}
              {}{\titlerule*[8pt]{.}\contentspage}
                                     % 添加section在目录里的点号
\usepackage{setspace}                % 调节行间距   
\usepackage{booktabs}                % 用于表格中加下划线
\usepackage{fancyhdr}                % 页眉页脚
\usepackage{type1cm}                 % 控制字体大小
\usepackage{indentfirst}             % 首行缩进
\usepackage{makeidx}                 % 建立索引
\usepackage{textcomp}                % 千分号等特殊符号
%\usepackage{layouts}                 % 打印当前页面格式
%\usepackage{bbding}                  % 一些特殊符号
\usepackage{cite}                    % 支持引用
\setlength{\skip\footins}{0.5cm}     % 脚注与正文的距离
%%%%%%%%%%%%%%%%%%%%%%%%%%以上是版面控制部分%%%%%%%%%%%%%%%%%%%%%%%%%%%%%%%%%%%%%%%%%%%%%

%%%%%%%%%%%%%%%%%%%%%%%%%%%%%%以下是一些命令或环境的重定义或自定义%%%%%%%%%%%%%%%%%%%%%%
\newtheorem{theorem}{定理}
\newtheorem{definition}{定义}
\newtheorem{property}{问题}
\newtheorem{proposition}{猜测}
\newtheorem{lemma}{引理}
\newtheorem{corollary}{推论}
\renewcommand{\contentsname}{\center\hei{\sanhao{目录}}}
\renewcommand{\refname}{\textbf{\xiaosi{\song{参考文献}}}}      % 将References改为参考文献
%%自定义环境
\newenvironment{chabstract}{{\hei{\xiaosi{摘要:}}}}%定义中文摘要

%%%%%%%%%%%%%%%%%%%%%%%%%%%%%%以上是一些命令或环境的重定义或自定义%%%%%%%%%%%%%%%%%%%%%%%%




%%%%%%%%%%%%%%%%%%%%%%%%%%%%%以下进入正文%%%%%%%%%%%%%%%%%%%%%%%%%%%%%%%%%%%%%%%%%%%%%%
\title{\hei{\sanhao{LATEX模板}}}
\author{\hei\xiaosi{吴鸿毅}} 
\date{\song 2014年4月30日} %添加时间
%%以上三个信息会在正文中有 \maketitle 的地方出现

\begin{document}

\begin{titlepage}   %封面
\mbox{}
\vspace*{20pt}%20pt的空白距离

%\centerline{\LARGE 综合××××}

\begin{figure}[!htb]%插入图片
  \centering
  \includegraphics[height=85pt]{tupianceshi.jpg}
\end{figure}

\vspace*{40pt}

\centerline{\song\fontsize{30pt}{40pt}\selectfont{本\quad \quad 科\quad \quad 毕\quad \quad 业\quad \quad 论\quad \quad 文}}

\vspace*{180pt}

\begin{spacing}{1.8}    %行间距增为1.8倍
\begin{center}
\begin{tabular}{rc}
\song\sanhao 论文题目:& \sanhao {\hei 暂时还不知道}\\ \cmidrule(r){2-2} %第二格加下划线
\song\sanhao 学生姓名:& \sanhao\hei 吴 \qquad 鸿 \qquad 毅\\ \cmidrule(r){2-2}
\song\sanhao 学\ \ \quad 号:& \sanhao\hei 2011151416\\ \cmidrule(r){2-2}
\song\sanhao 专\ \ \quad 业: & \sanhao\hei 核~工~程~与~核~技~术 \\ \cmidrule(r){2-2}
\song\sanhao 指导老师: & \sanhao\hei 无 \quad 名 \quad 氏\\ \cmidrule(r){2-2}
\song\sanhao 学\ \ \quad 院: & \sanhao\hei 核~科~学~与~技~术~学~院\\ \cmidrule(r){2-2}
\end{tabular}
\end{center}
\end{spacing}

\vfill

\centerline{\hei\normalsize 哈尔滨工程大学}
\centerline{\song\sanhao 2014年4月30日}

\end{titlepage}
\thispagestyle{empty}% 首页无页眉页脚
\newpage%另起一页

\begin{spacing}{2}
\tableofcontents                    % 插入目录,需要编译两次才能出现
\end{spacing}
\thispagestyle{empty}               % 首页无页眉页脚

%%%%%%%%%%%%%%%%%%%%%%%%%%%%%以下是中文摘要、关键词%%%%%%%%%%%%%%%%%%%%%%%%%%%%%%%%%%%%
\clearpage %双面打印(openright) 用\cleardoublepage,刷新页面信息,为了添加目录章节后页码不乱
\pagenumbering{arabic} %自此处页码开始计数       
\addcontentsline{toc}{section}{\textbf{\song\xiaosi{摘要}}} %创建虚拟章节,便于将摘要部分添加到目录
\maketitle
\begin{chabstract}
\hei 本文讨论了本科层次的泛函分析教材中函数空间和数列空间的实例,$\cdots$.\\
\end{chabstract}


%%%%%%%%%%%%%%%%%%%%%%%%%%%%%以上是中文摘要、关键词%%%%%%%%%%%%%%%%%%%%%%%%%%%%%%%%%


%%%%%%%%%%%%%%%%%%%%%%%%%%%%%以下为论文引言部分%%%%%%%%%%%%%%%%%%%%%%%%%%%%%%%%%%%%%%%%%
\section{\textbf{\song\xiaosi{引言}}}

{\xiaosi{\song{在一般的本科生泛函分析教材中, $\cdots$ 如~$L^p(\mathbf{E})$~和~$l^p$~}}}\\
%%%%%%%%%%%%%%%%%%%%%%%%%%%%%以上为论文引言部分%%%%%%%%%%%%%%%%%%%%%%%%%%%%%%%%%%%%%%%%%%


{\xiaosi{\song{
%%%%%%%%%%%%%%%%%%%%%%%%%%%%%以下为论文第二部分%%%%%%%%%%%%%%%%%%%%%%%%%%%%%%%%%%%%%%%%%%
\section{\textbf{\song\xiaosi 数列空间和函数空间的定义}}

以下给出六种典型的数列空间和函数空间的定义,文字叙述和符号表示依照文献\cite{cankaowenxiao111}.\\

\begin{definition}[空间~$l^p$   $(p\geq 1)$]               %%%%%%%%%%%%%%定义 1%%%%%%%%%%%
一切满足
~$(\sum\limits^{\infty}_{i=1}|\xi_i|^p)^{1/p}<+\infty$~    %注意命令 \limits
的数列~$x=(\xi_1,\xi_2,\cdots)$~的全体记为~$l^p$.\ 容易验证
$${\parallel x\parallel}_{p}=(\sum^{\infty}_{i=1}|\xi_i|^p)^{1/p}<+\infty$$ 是~$l^p$~上的范数.
\end{definition}
$\cdots$
%%%%%%%%%%%%%%%%%%%%%%%%以上为论文正文第二部分%%%%%%%%%%%%%%%%%%%%%%%%%%%%%%%%%%%%%%%%%%%%%%%


%%%%%%%%%%%%%%%%%%%%%%%%以下为论文正文第三部分%%%%%%%%%%%%%%%%%%%%%%%%%%%%%%%%%%%%%%%%%%%%%%
\section{\textbf{\song{\xiaosi 六类空间各自的性质}}}

{\subsection{\textbf{\song{\xiaosi{$l^p$和$L^p(\mathbf{E})$}}}}}
$l^p$~和~$L^p(\mathbf{E})$~都可分.
$\cdots$
%%%%%%%%%%%%%%%%%%%%%%%%%%%以上为论文正文第三部分%%%%%%%%%%%%%%%%%%%%%%%%%%%%%%%%%%%%%%%%%%


%%%%%%%%%%%%%%%%%%%%%%%%%%%以下为论文正文第四部分%%%%%%%%%%%%%%%%%%%%%%%%%%%%%%%%%%%%%%%%%%
{\section{\textbf{\song{\xiaosi 六种空间之间的一些联系}}}}

{\subsection{\textbf{\song\xiaosi{函数空间与函数空间、数列空间与数列空间之间的联系}}}}


$\cdots$
{\subsection{\textbf{\song\xiaosi{函数空间与数列空间的联系}}}}

\begin{lemma}[~Riesz-Fiesher~定理]                %%%%%%%%%%%%%%%%%%%%%% 引理 1 %%%%%%%%%%%%%%%%%%%%%%%%
设~$\{e_n\}$~是~Hilbert~空间~$\mathbf{H}$~中一就范正交系,$(c_1,c_2,\cdots)\in l^2$,则存在唯一的~$x\in H$~使~$(x,e_n)=e_n,\quad n=1,2,\cdots$~并且~$(x,x)=\sum\limits_{n=1}^\infty|c_n|^2$.\\
\end{lemma}


文献\cite{cankao4}给出了~$1\leq p\leq 2$~时的~$L^p(\mathbf{E})$~上的~Fourier~变换的构造过程,并指出当
~$p>2$~时在广义函数的意义下~$L^p(\mathbf{E})$~仍可导入~Fourier~变换.\ 问题在于~$p\neq 2$~时~Fourier
~变换能否构成~$L^p(\mathbf{E})$~与~$l^p$~之间的保范同构.


\begin{property}\label{pro2}                  %%%%%%%%%%%%%%%%% 问题 2 %%%%%%%%%%%%%%%%%%%%%%%%
完备距离空间~$S(\mathbf{E})$~与~$s$,Banach~空间~$M(\mathbf{E})$~与~$m$~之间是否有同构关系?更进一步,~Fourier~变换及其反演公式
能否推广到完备距离空间~$S(\mathbf{E})$~与~$s$,Banach~空间~$M(\mathbf{E})$~ 与~$m$
\end{property}
%%%%%%%%%%%%%%%%%%%%%%%%%以上是论文正文的第四部分%%%%%%%%%%%%%%%%%%%%%%%%%%%%%%%%%%%%%%%%%%%%
}}}


%%%%%%%%%%%%%%%%%%%%%%%%%以下是参考文献%%%%%%%%%%%%%%%%%%%%%%%%%%%%%%%%%%%%%%%%%%%%%%%%
\clearpage %双面打印(openright) 用\cleardoublepage
%\addcontentsline{toc}{section}{\textbf{\hei\xiaosi{参考文献}}}
\begin{thebibliography}{99}
{\song{\wuhao
\bibitem{cankaowenxiao111}那汤松.\ 实变函数论(第5版).\ 徐瑞云 译.\ 北京:高等教育出版社,2010.
%\bibitem{cankao2}郭大钧等.\ 实变函数与泛函分析(第二版)$\cdot$ 下册.\ 山东:山东大学出版社,2005.
%\bibitem{3}夏道行等.\ 实变函数论与泛函分析(下册$\cdot$ 第二版修订本).\ 北京:高等教育出版社,2010.
\bibitem{cankao4}A.H.柯尔莫戈洛夫,C.B.佛明.\ 函数论与泛函分析初步(第7 版).\ 北京:高等教育出版社,2006.
    }}
\end{thebibliography}


\end{document}