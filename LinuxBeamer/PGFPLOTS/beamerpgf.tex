%%% beamerpgf.tex --- 
%% 
%% Description: 
%% Author: Hongyi Wu(吴鸿毅)
%% Email: wuhongyi@qq.com 
%% Created: 五 10月 24 20:58:05 2014 (+0800)
%% Last-Updated: 四 7月 30 20:43:58 2015 (+0800)
%%           By: Hongyi Wu(吴鸿毅)
%%     Update #: 105
%% URL: http://wuhongyi.cn 

\documentclass[presentation]{beamer}

\usepackage{fontspec}
\setsansfont{WenQuanYi Zen Hei}%Linux下使用文泉驿正黑或其他系统中可用的字体
\XeTeXlinebreaklocale "zh"  % 表示用中文的断行
\XeTeXlinebreakskip = 0pt plus 1pt % 多一点调整的空间
\setlength{\parindent}{2em}%设置缩进为两个大写M的宽度,大约为两个汉字的宽度

\usepackage{graphics}
\usepackage{xunicode}
\usepackage{xltxtra}
\usepackage{beamerthemesplit}
\usepackage{hyperref}%为PDF文档创建超链接
\usepackage{amsmath}%数学符号与公式
\usepackage{amsfonts} %数学符号与字体
\usepackage{supertabular}
\usepackage{multirow} %合并表格列
\usepackage{tabularx} %调节表格宽度

\usetheme{CambridgeUS}%深红色,好看

%%pgf
\usepackage{pgfplotstable}
\usepackage{tikz}

\title{吴鸿毅beamer模板}
\subtitle{Beamer pgfplots}
\author{吴鸿毅}
\institute{哈尔滨工程大学 核科学与技术学院}
\date{\today}
\logo{\includegraphics[height=1cm, width=1cm]{logo.jpeg}}

\begin{document}

\begin{frame}
\titlepage
\end{frame}


\begin{frame}
\frametitle{目录}
    \tableofcontents     %你也可以插入选项 [pausesections]
\end{frame}

\section{beamer}
\subsection{PGF PLOTS}

\begin{frame}
  \frametitle{3.2.3}%帧标题
  \begin{tikzpicture}
    \begin{axis}[
      ]
      % density of Normal distribution:
      \addplot[
        red,
        domain=-3e-3:3e-3,
        samples=201,
      ]
              {exp(-x^2 / (2e-3^2)) / (1e-3 * sqrt(2*pi))};
    \end{axis}
  \end{tikzpicture}
\end{frame}

\begin{frame}
  \frametitle{3.2.5}
  \begin{tikzpicture}
    \begin{axis}[
        axis lines=left,
        scaled ticks=false,
        xticklabel style={
          rotate=90,
          anchor=east,
          /pgf/number format/precision=3,
          /pgf/number format/fixed,
          /pgf/number format/fixed zerofill,
        },
      ]
      % density of Normal distribution:
      \newcommand\MU{0}
      \newcommand\SIGMA{1e-3}
      \addplot[red,domain=-3*\SIGMA:3*\SIGMA,samples=201]
              {exp(-(x-\MU)^2 / 2 / \SIGMA^2)
                / (\SIGMA * sqrt(2*pi))};
    \end{axis}
  \end{tikzpicture}
\end{frame}

\begin{frame}
  \frametitle{3.3.1}%帧标题
  % Preamble: \pgfplotsset{width=7cm,compat=1.11}
  %\pgfplotsset{compat=1.5}
  \begin{tikzpicture}
    \begin{loglogaxis}[
        title=Convergence Plot,
        xlabel={Degrees of freedom},
        ylabel={$L_2$ Error},
      ]
      \addplot table {data_d2.dat};
    \end{loglogaxis}
  \end{tikzpicture}
\end{frame}

\begin{frame}
  \frametitle{3.3.2}%帧标题
  % Preamble: \pgfplotsset{width=7cm,compat=1.11}
  \begin{tikzpicture}
    \begin{loglogaxis}[
        title=Convergence Plot,
        xlabel={Degrees of freedom},
        ylabel={$L_2$ Error},
      ]
      \addplot table {data_d2.dat};
      \addplot table {data_d3.dat};
      \addplot table {data_d4.dat};
    \end{loglogaxis}
  \end{tikzpicture}
\end{frame}

\begin{frame}
  \frametitle{3.3.3}%帧标题
  \begin{tikzpicture}
    \begin{loglogaxis}[
        title=Convergence Plot,
        xlabel={Degrees of freedom},
        ylabel={$L_2$ Error},
        grid=major,
        legend entries={$d=2$,$d=3$,$d=4$},
      ]
      \addplot table {data_d2.dat};
      \addplot table {data_d3.dat};
      \addplot table {data_d4.dat};
    \end{loglogaxis}
  \end{tikzpicture}
\end{frame}

\begin{frame}
  \frametitle{3.3.4}%帧标题
  \begin{tikzpicture}
    \begin{loglogaxis}[
        title=Convergence Plot,
        xlabel={Degrees of freedom},
        ylabel={$L_2$ Error},
        grid=major,
        legend entries={$d=2$,$d=3$,$d=4$},
      ]
      \addplot table {data_d2.dat};
      \addplot table {data_d3.dat};
      \addplot table {data_d4.dat};
      \addplot table[
        x=dof,
        y={create col/linear regression={y=l2_err,
            variance list={1000,800,600,500,400,200,100}}}]
               {data_d4.dat};
    \end{loglogaxis}
  \end{tikzpicture}
\end{frame}

\begin{frame}
  \frametitle{3.3.5-1}%帧标题
  \begin{tikzpicture}
    \begin{loglogaxis}[
        title=Convergence Plot,
        xlabel={Degrees of freedom},
        ylabel={$L_2$ Error},
        grid=major,
        legend entries={$d=2$,$d=3$,$d=4$},
      ]
      \addplot table {data_d2.dat};
      \addplot table {data_d3.dat};
      \addplot table {data_d4.dat};
      \addplot table[
        x=dof,
        y={create col/linear regression={y=l2_err,
            variance list={1000,800,600,500,400,200,100}}}]
               {data_d4.dat}
               % save two points on the regression line
               % for drawing the slope triangle
               coordinate [pos=0.25] (A)
               coordinate [pos=0.4] (B)
               ;
               % save the slope parameter:
               \xdef\slope{\pgfplotstableregressiona}
               % draw the opposite and adjacent sides
               % of the triangle
               \draw (A) -| (B)
               node [pos=0.75,anchor=west]
               {\pgfmathprintnumber{\slope}};
    \end{loglogaxis}
  \end{tikzpicture}
\end{frame}

%% \begin{frame}
%%   \frametitle{3.3.5-2}%帧标题
%%   \begin{tikzpicture}
%%     \begin{loglogaxis}
%%       \addplot table {data_d2.dat}
%%       coordinate [pos=0.25] (A)
%%       coordinate [pos=0.4] (B)
%%       ;
%%       \draw[-stealth] (A) -| (B);
%%       \node[pin=0:Special.] at (769,1.623e-04) {};
%%     \end{loglogaxis}
%%   \end{tikzpicture}
%% \end{frame}

\begin{frame}
  \frametitle{3.3.5-3}%帧标题
  \begin{tikzpicture}
    \begin{loglogaxis}[
        tiny,
      ]
      \addplot table {data_d2.dat}
      node [pos=1,pin=0:Special.] {}
      ;
    \end{loglogaxis}
  \end{tikzpicture}
\end{frame}

\begin{frame}
  \frametitle{3.3.5-4}%帧标题
  \begin{tikzpicture}
    \begin{loglogaxis}[
        tiny,
        clip=false,
      ]
      \addplot table {data_d2.dat}
      node [pos=1,pin=0:Special.] {}
      ;
    \end{loglogaxis}
  \end{tikzpicture}
\end{frame}

%% \begin{frame}
%%   \frametitle{3.4.2}%帧标题
%%   \begin{tikzpicture}
%%     \begin{axis}
%%       \addplot[
%%         scatter,
%%         only marks,
%%         point meta=explicit symbolic,
%%         scatter/classes={
%%           a={mark=square*,blue},%
%%           b={mark=triangle*,red},%
%%           c={mark=o,draw=black}},
%%       ]
%%       table[meta=label] {
%%         x      y      label
%%         0.1    0.15   a
%%         0.45   0.27   c
%%         0.02   0.17   a
%%         0.06   0.1    a
%%         0.9    0.5    b
%%         0.5    0.3    c
%%         0.85   0.52   b
%%         0.12   0.05   a
%%         0.73   0.45   b
%%         0.53   0.25   c
%%         0.76   0.5    b
%%         0.55   0.32   c
%%       };
%%     \end{axis}
%%   \end{tikzpicture}
%% \end{frame}

%% \begin{frame}
%%   \frametitle{3.4.3}%帧标题
%% \begin{tikzpicture}
%% \begin{axis}[
%% enlargelimits=0.2,
%% ]
%% \addplot+[nodes near coords,only marks,
%% point meta=explicit symbolic]
%% table[meta=label] {
%% x   y   label
%% 0.5 0.2 1
%% 0.2 0.1 t2
%% 0.7 0.6 3
%% 0.35 0.4 Y4
%% 0.65 0.1 5
%% };
%% \end{axis}
%% \end{tikzpicture}
%% \end{frame}

\begin{frame}
  \frametitle{3.5.4-1}%帧标题
% Preamble: \pgfplotsset{width=7cm,compat=1.11}
\begin{tikzpicture}
\begin{axis}[
title={$x \exp(-x^2-y^2)$},
xlabel=$x$, ylabel=$y$,
small,
]
\addplot3
{exp(-x^2-y^2)*x};
\end{axis}
\end{tikzpicture}

\end{frame}

\begin{frame}
  \frametitle{3.5.4-2}%帧标题
\begin{tikzpicture}
\begin{axis}[
title={$x \exp(-x^2-y^2)$},
xlabel=$x$, ylabel=$y$,
small,
]
\addplot3[
surf,
domain=-2:2,
domain y=-1.3:1.3,
]
{exp(-x^2-y^2)*x};
\end{axis}
\end{tikzpicture}
\end{frame}

%% \begin{frame}
%%   \frametitle{3.5.4-3}%帧标题
%% \begin{tikzpicture}
%% \begin{axis}[
%% title={$x \exp(-x^2-y^2)$},
%% xlabel=$x$, ylabel=$y$,
%% small,
%% ]
%% \addplot3[
%% contour gnuplot,
%% domain=-2:2,
%% domain y=-1.3:1.3,
%% ]
%% {exp(-x^2-y^2)*x};
%% \end{axis}
%% \end{tikzpicture}
%% \end{frame}

%% \begin{frame}
%%   \frametitle{3.5.4-4}%帧标题
%% \begin{tikzpicture}
%% \begin{axis}[
%% title={$x \exp(-x^2-y^2)$},
%% enlarge x limits,
%% view={0}{90},
%% xlabel=$x$, ylabel=$y$,
%% small,
%% ]
%% \addplot3[
%% domain=-2:2,
%% domain y=-1.3:1.3,
%% contour gnuplot={number=14},
%% thick,
%% ]
%% {exp(-x^2-y^2)*x};
%% \end{axis}
%% \end{tikzpicture}
%% \end{frame}

\begin{frame}
  \frametitle{}%帧标题

\end{frame}

\begin{frame}
  \frametitle{}%帧标题

\end{frame}

\begin{frame}
  \frametitle{}%帧标题

\end{frame}

\begin{frame}
  \frametitle{}%帧标题

\end{frame}

\begin{frame}
  \frametitle{}%帧标题

\end{frame}

\begin{frame}
  \frametitle{}%帧标题

\end{frame}

\begin{frame}
  \frametitle{}%帧标题

\end{frame}

\begin{frame}
  \frametitle{}%帧标题

\end{frame}

\begin{frame}
  \frametitle{}%帧标题

\end{frame}

\begin{frame}
  \frametitle{}%帧标题

\end{frame}

\begin{frame}
  \frametitle{}%帧标题

\end{frame}

\begin{frame}
  \frametitle{}%帧标题
\begin{tikzpicture}
\path (0cm,0cm) coordinate(O);
\draw (O) circle(3cm);
\draw (O)++(0.1cm,-0.8cm) arc(45:315:0.2cm);
\path ++(-1.5cm,0.5cm) coordinate (P1) ++(3cm,0cm) coordinate (P2);
\foreach \i in {1,2}{
\draw (P\i) ellipse(1cm and 0.5cm);
\fill[black] (P\i) circle (0.5cm);
\fill[gray] (P\i) ellipse (0.125cm and 0.25cm);
}
\begin{scope}
\clip (O)++(0,-1.8cm) circle (0.7cm);
\fill[color=gray](-3cm,-3cm) rectangle (2cm,-2cm);
\end{scope}
\end{tikzpicture}
\end{frame}








\end{document}


%%%%%%%%%%%%%%%%%%
\begin{frame}[]
\begin{frame}
  \frametitle{}%帧标题
  %\framesubtitle{}%帧副题

\end{frame}
其中可选参数有下列选项:
allowframebreaks 当帧环境中的内容过多,超出一幅幻灯片所能显示的范围时,超出部分就看不到,如果使用该选项,帧环境就能自动换幅,即自动增加一幅以放置超出的内容。
allowdisplaybreaks 允许多行公式中间换幅,该选项必须与allowframebreaks选项同时使用,否则无效。
t 每幅幻灯片中的内容顶对齐。
c 默认值,即每幅幻灯片中的内容垂直居中。帧环境的t、c选项优先于beamer的t、c选项。
b 每幅幻灯片中的内容底对齐。
fragile 它告诉beamer帧环境中的内容是“脆弱”的,不能按通常的意义来编译,例如在使用抄录环境verbatim时就要添加此选项。
containsverbatim 使用抄录环境verbatim或\verb命令时要添加此选项。
squeeze 压缩文本行之间的行距。
shrink 帧环境中的内容超出一幅幻灯片所能显示的范围时,超出部分就看不到了,如果使用该选项,可自动缩小帧环境中所有内容的字体尺寸,并压缩行距,使帧环境中的全部内容都能够放在一幅幻灯片里。使用本选项时,squeeze选项也就自动被启用了。
plain 取消各种导航条和微标(logo),以便创建其他样式的导航条或是显示一张满幅的插图等。微标是用\logo命令生成的插图,通常是校徽或是会徽。
%%%%%%%%%%%%%%%%%%
\beamersetaveragebackground{black!10}%设置帧背景德文颜色

\alert{}%将其中的文本字体改为红色,它的作用类似于\emph命令,用于强调某一词语。


\begin{tabular}{|c|c|c|c|c|}%表格合并行列
\hline \hline 
lable 1-1 & label 1-2 & label 1-3 & label 1 -4 & label 1-5\\\hline
label 2-1 & label 2-2 & label 3-3 & label 4-4 & label 5-5 \\\hline
\multirow{2}{*}{Multi-Row} & \multicolumn{2}{|c|}{Multi-Column} & \multicolumn{2}{|c|}{\multirow{2}{*}{Multi-Row and Col}} \\\cline{2-3} 
& column-1 & column-2 & \multicolumn{2}{|c|}{}\\%补偿上面的两列合并的那一行
\hline
\end{tabular}


%%%%%%%%%%%%%%%%%%%%%%%%%%%%%%%%%%%%%%%%%%%%%%%%%%%%%%%%%%%%%%%%%%%%%%
%%% beamerpgf.tex ends here
