%%% beamerpfg.tex --- 
%% 
%% Description: 
%% Author: Hongyi Wu(吴鸿毅)
%% Email: wuhongyi@qq.com 
%% Created: 五 10月 24 20:58:05 2014 (+0800)
%% Last-Updated: 日 3月 15 20:26:34 2015 (+0800)
%%           By: Hongyi Wu(吴鸿毅)
%%     Update #: 107
%% URL: http://wuhongyi.cn 

\documentclass[presentation]{beamer}

\usepackage{fontspec}
\setsansfont{WenQuanYi Zen Hei}%Linux下使用文泉驿正黑或其他系统中可用的字体
\XeTeXlinebreaklocale "zh"  % 表示用中文的断行
\XeTeXlinebreakskip = 0pt plus 1pt % 多一点调整的空间
\setlength{\parindent}{2em}%设置缩进为两个大写M的宽度,大约为两个汉字的宽度

\usepackage{graphics}
\usepackage{xunicode}
\usepackage{xltxtra}
\usepackage{beamerthemesplit}
\usepackage{hyperref}%为PDF文档创建超链接
\usepackage{amsmath}%数学符号与公式
\usepackage{amsfonts} %数学符号与字体
\usepackage{supertabular}
\usepackage{multirow} %合并表格列
\usepackage{tabularx} %调节表格宽度

\usetheme{CambridgeUS}%深红色,好看

%%table
\usepackage{graphicx}


\title{吴鸿毅beamer模板}
\subtitle{Beamer table}
\author{吴鸿毅}
\institute{哈尔滨工程大学 核科学与技术学院}
\date{\today}
\logo{\includegraphics[height=1cm, width=1cm]{logo.jpeg}}

\begin{document}

\begin{frame}
\titlepage
\end{frame}


\begin{frame}
\frametitle{目录}
    \tableofcontents     %你也可以插入选项 [pausesections]
\end{frame}

\section{beamer table}
\subsection{table}

\begin{frame}
  \frametitle{生成固定宽度的表格}
\begin{tabular*}{5cm}{@{\extracolsep{\fill}}lllr}
\hline
1 & 2 & 3 & 4 \\
21 & 22 & 23 & 24 \\
\hline
\end{tabular*}

这种方法的一个缺点是列之间如果加入分隔符,则列之间的空隙是放在下一列的左边而不是在两列之间平均分配。我们来看上面的例子加入分隔符后的样子

\begin{tabular*}{5cm}
{@{\extracolsep{\fill}}|l|l|l|r|}
\hline
1 & 2 & 3 & 4 \\
21 & 22 & 23 & 24 \\
\hline
\end{tabular*}
\end{frame}

\begin{frame}
  \frametitle{tabular 环境插入表格}
\begin{tabular}{|l|c|r|}
\hline
左列 & 中列 & 右列 \\
\hline
第二行 & 第二行 & 第二行 \\
\hline
第三行 & 第三行 & 第三行 \\
\hline
第四行 & 第四行 & 第四行 \\
\hline
\end{tabular}

首先,tabular 环境的参数 |l|c|r| 指明了各列的对齐方式,l、c 和 r 分别表示左对齐、居中对齐和右对齐。中间的竖线 | 指明各列之间有竖线分隔,如果在某些地方不需要竖线,去掉相应位置的 | 即可。

表格各行的元素之间用 \& 号分隔,两行内容用 $\backslash$$\backslash$ 分隔。$\backslash$hline 表示两行之间的横线;你可以用连续两个 $\backslash$hline 得到双横线,或者去掉 $\backslash$hline 以不显示该横线。

如果需要在某个单元格中填写多行内容,不能直接用 $\backslash$$\backslash$ 或 $\backslash$newline 命令,而应该将它们放在一个盒子里面(比如 $\backslash$parbox 盒子)。

\end{frame}

\begin{frame}
  \frametitle{跨列表格}%帧标题
  %\framesubtitle{}%帧副题
复杂的表格经常需要跨行和跨列,在 tabular 环境中,我们可以用命令 $\backslash$multicolumn 得到跨列表格,而跨行表格需要使用 multirow 宏包。
\begin{tabular}{|l|c|r|}
\hline
左列 & 中列 & 右列 \\
\hline
第二行 & 第二行 & 第二行 \\
\hline
\multicolumn{2}{|c|}{跨越2011} & 第三行 \\
\hline
第四行 & 第四行 & 第四行 \\
\hline
\end{tabular}

上面的 $\backslash$multicolumn 命令的第一个参数指明要横跨的列数,第二个参数指明对齐和边框线,第三个参数指明该单元格的内容。

\end{frame}



\begin{frame}
  \frametitle{浮动表格}%帧标题
  %\framesubtitle{}%帧副题
  表格是在 tabular 环境对应的位置排版出来的。如果表格高度大于当前页剩余高度,表格就会被放置到下一页中,造成这一页下部留出很大空白。大部分时候我们并不需要严格限定表格出现的位置,而只要求表格在该段正文的附近出现即可。此时,我们可以用 table 浮动环境来达到自动调整位置的效果。

  \begin{table}[htbp!]
  \centering
  \begin{tabular}{|l|c|r|c|}
    \hline
    第一行 & 第一行 & 第一行 & 第一行 \\
    \hline
    第二行 & 第二行 & 第二行 & 第二行 \\
    \hline
    第三行 & 第三行 & 第三行 & 第三行\\
    \hline
    第四行 & 第四行 & 第四行 & 第四行 \\
    \hline
  \end{tabular}
  \end{table}

其中的可选参数里,h(here,当前位置)、t(top,页面顶部)、b(bottom,页面底部)、p(page,单独一页)表明允许将表格放置在哪些位置,而 ! 表示不管某些浮动的限制。用 table 浮动环境,还可以用 $\backslash$caption命令指明表格的名称,并得到表格的自动编号。

\end{frame}



\begin{frame}[allowframebreaks]
  \frametitle{无框线表格环境 tabbing}%帧标题
它没有表格框线绘制命令,列与列之间采用空格分隔,列数据处于左右模式中,不能自动换行。
在 tabbing 表格环境中可使用以下多种制表命令。

$\backslash$=  列宽命令,表示两列之间以此为界,第一行的各列必须用其确定列宽度,一列的宽度是由其第 1 行数据的自然宽度加所设水平空白宽度;该命令还可用在其它行来设置新的列。

$\backslash$$\backslash$ 换行命令,表示当前行结束,开始心的一行。最后一行可省略此命令。可采用如$\backslash$$\backslash$[5pt]的方法加宽与下一行的距离。

$\backslash$>  分列命令,用于分隔两列数据。

$\backslash$+  右移命令,表示将其后各行数据右移动一列。

$\backslash$-  左移命令,表示将其后各行数据向左移动一列,相当于取消之前设置的一个 $\backslash$+ 的作用。

$\backslash$<  用于一行之首,表示该行数据向左移动一列,相当于取消之前设置的一个 $\backslash$+ 在该行的作用。

$\backslash$‘  可用于一行中最后一列数据之前的任何位置,它可将该行最后一列数据移至版心右侧边。

$\backslash$’  该命令可以插在某一数据之中,将其一分为二,左半部置于该列左侧的左边,右半部置于右边;左半部与该列左侧之间的距离可以用 $\backslash$tabbingsep 长度数据命令来设定,其默认值为 0.5em。

$\backslash$a  在 LATEX 中,命令 $\backslash$=、$\backslash$‘ 和 $\backslash$’ 是用于生成重音字符的,例如 $\backslash$=o、$\backslash$‘o 和 $\backslash$'o ,生成 \=o 、 \'o 和 \'o ;tabbing 环境重新定义了这三个命令,如果在表格数据中有这三种重音符号,可改用 $\backslash$a=、$\backslash$a’ 和 $\backslash$a’ 来生成。(有误)

$\backslash$kill  取消命令,它用于取代换行命令 $\backslash$$\backslash$ ,表示取消该行,即该行的内容不被排版出来,但其中的 $\backslash$= 、 $\backslash$+ 、 或$\backslash$- 命令和列宽度设置仍然有效。

$\backslash$pushtabs 堆栈命令,存储当前各列宽度的设置。

$\backslash$poptabs 弹出命令,恢复最后一个 $\backslash$pushtabs 命令所存储的各列宽度设置。

下面是使用 tabbing 环境编排一个四列无框线表格:
\begin{tabbing}
项目 \hspace{4mm} \= {\sf 802.11b} \hspace{4mm} \= 蓝牙 \hspace{10mm} \= {\sf HomeRF}\\
频率 \> 2.4GHz \> 2.4GHz \> 2.4GHz\\
技术 \> DSSS \> FHSS \> FHSS\\
\end{tabbing}

下面是左移和右移命令在 tabbing 表格环境中的应用:
\begin{tabbing}
夸克quark \hspace{5mm}\= 介子meson\\
\> 重子baryon \hspace{5mm}\= 核子nucleon \+\+\\
超子hyperon \-\\
轻子 \> 强子 \\
\< 量子 \> 光子 \> 胶子\\
引力子 \> 重子 \\
\end{tabbing}

在本例中 $\backslash$+$\backslash$+ 表示将其后各行的数据右移两列。
\end{frame}


\begin{frame}
  \frametitle{四周带边线表格 tabular}%帧标题
  %\framesubtitle{}%帧副题
\
\begin{tabular}{|l|c|r|}\hline
\multicolumn{3}{|c|}{Sample Tabular}\\\hline
col head & col head & col head\\\hline
Left & centered & right\\\cline{1-2}
aligned & items & aligned\\\cline{2-3}
items & items & items\\\cline{1-2}
Left items & centered & right\\\hline
\end{tabular}
\end{frame}


\begin{frame}
  \frametitle{小数点对齐 tabular}%帧标题
  %\framesubtitle{}%帧副题
  \begin{tabular}{c r @{.} l}\hline
    太阳系中的行星 & \multicolumn{2}{c}{赤道半径km}\\\hline
    水星 & 2 & 44\\
    金星 & 6 & 1\\
    地球 & 6378 & 142\\\hline
  \end{tabular}

在本例中,用 @ 表达式:@\{.\},将小数点插在数列与小数之间,同时将这两列之间的空白删除;用 $\backslash$multicolumn 命令将表格第 1 行的第 2 列和第 3 列合并为一列,并将该列数据居中,成为整数列和小数列共有的列标题。
\end{frame}


\begin{frame}
  \frametitle{}%帧标题
  %\framesubtitle{}%帧副题

\end{frame}


\begin{frame}
  \frametitle{}%帧标题
  %\framesubtitle{}%帧副题

\end{frame}














\begin{frame}
  \frametitle{本页标题}

\begin{tabular}{|c|c|c|c|c|}%表格合并行列
\hline \hline 
lable 1-1 & label 1-2 & label 1-3 & label 1 -4 & label 1-5\\\hline
label 2-1 & label 2-2 & label 3-3 & label 4-4 & label 5-5 \\\hline
\multirow{2}{*}{Multi-Row} & \multicolumn{2}{|c|}{Multi-Column} & \multicolumn{2}{|c|}{\multirow{2}{*}{Multi-Row and Col}} \\\cline{2-3} 
& column-1 & column-2 & \multicolumn{2}{|c|}{}\\%补偿上面的两列合并的那一行
\hline
\end{tabular}

\end{frame}




\end{document}


%%%%%%%%%%%%%%%%%%
\begin{frame}[]
\begin{frame}
  \frametitle{}%帧标题
  %\framesubtitle{}%帧副题

\end{frame}
其中可选参数有下列选项:
allowframebreaks 当帧环境中的内容过多,超出一幅幻灯片所能显示的范围时,超出部分就看不到,如果使用该选项,帧环境就能自动换幅,即自动增加一幅以放置超出的内容。
allowdisplaybreaks 允许多行公式中间换幅,该选项必须与allowframebreaks选项同时使用,否则无效。
t 每幅幻灯片中的内容顶对齐。
c 默认值,即每幅幻灯片中的内容垂直居中。帧环境的t、c选项优先于beamer的t、c选项。
b 每幅幻灯片中的内容底对齐。
fragile 它告诉beamer帧环境中的内容是“脆弱”的,不能按通常的意义来编译,例如在使用抄录环境verbatim时就要添加此选项。
containsverbatim 使用抄录环境verbatim或\verb命令时要添加此选项。
squeeze 压缩文本行之间的行距。
shrink 帧环境中的内容超出一幅幻灯片所能显示的范围时,超出部分就看不到了,如果使用该选项,可自动缩小帧环境中所有内容的字体尺寸,并压缩行距,使帧环境中的全部内容都能够放在一幅幻灯片里。使用本选项时,squeeze选项也就自动被启用了。
plain 取消各种导航条和微标(logo),以便创建其他样式的导航条或是显示一张满幅的插图等。微标是用\logo命令生成的插图,通常是校徽或是会徽。
%%%%%%%%%%%%%%%%%%
\beamersetaveragebackground{black!10}%设置帧背景德文颜色

\alert{}%将其中的文本字体改为红色,它的作用类似于\emph命令,用于强调某一词语。


\begin{tabular}{|c|c|c|c|c|}%表格合并行列
\hline \hline 
lable 1-1 & label 1-2 & label 1-3 & label 1 -4 & label 1-5\\\hline
label 2-1 & label 2-2 & label 3-3 & label 4-4 & label 5-5 \\\hline
\multirow{2}{*}{Multi-Row} & \multicolumn{2}{|c|}{Multi-Column} & \multicolumn{2}{|c|}{\multirow{2}{*}{Multi-Row and Col}} \\\cline{2-3} 
& column-1 & column-2 & \multicolumn{2}{|c|}{}\\%补偿上面的两列合并的那一行
\hline
\end{tabular}


%%%%%%%%%%%%%%%%%%%%%%%%%%%%%%%%%%%%%%%%%%%%%%%%%%%%%%%%%%%%%%%%%%%%%%
%%% beamerpfg.tex ends here
