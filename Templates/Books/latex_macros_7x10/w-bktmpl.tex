%%%%%%%%%%%%%%
%% Run LaTeX on this file several times to get Table of Contents,
%% cross-references, and citations.

%% w-bktmpl.tex. Current Version: Feb 16, 2012
%%%%%%%%%%%%%%%%%%%%%%%%%%%%%%%%%%%%%%%%%%%%%%%%%%%%%%%%%%%%%%%%
%
%  Template file for
%  Wiley Book Style, Design No.: SD 001B, 7x10
%  Wiley Book Style, Design No.: SD 004B, 6x9
%
%  Prepared by Amy Hendrickson, TeXnology Inc.
%  http://www.texnology.com
%%%%%%%%%%%%%%%%%%%%%%%%%%%%%%%%%%%%%%%%%%%%%%%%%%%%%%%%%%%%%%%%

%%%%%%%%%%%%%%%%%%%%%%%%%%%%%%%%%%%%%%%%%%%%%%%%%%%%%%%%%%%%%%%%
%% Class File

%% For default 7 x 10 trim size:
\documentclass{WileySev}

%% Or, for 6 x 9 trim size
%\documentclass{WileySix}

%%%%%%%%%%%%%%%%%%%%%%%%%%%%%%%%%%%%%%%%%%%%%%%%%%%%%%%%%%%%%%%%
%% Post Script Font File

% For PostScript text
% If you have font problems, you may edit the w-bookps.sty file
% to customize the font names to match those on your system.

\usepackage{w-bookps}

%%%%%%%
%% For times math: However, this package disables bold math (!)
%% \mathbf{x} will still work, but you will not have bold math
%% in section heads or chapter titles. If you don't use math
%% in those environments, mathptmx might be a good choice.

% \usepackage{mathptmx}


%%%%%%%%%%%%%%%%%%%%%%%%%%%%%%%%%%%%%%%%%%%%%%%%%%%%%%%%%%%%%%%%
%% Graphicx.sty for Including PostScript .eps files

\usepackage{graphicx}

%%%%%%%%%%%%%%%%%%%%%%%%%%%%%%%%%%%%%%%%%%%%%%%%%%%%%%%%%%%%%%%%
%% Other packages you might want to use:

% for chapter bibliography made with BibTeX
% \usepackage{chapterbib}

% for multiple indices
% \usepackage{multind}

% for answers to problems
% \usepackage{answers}

%%%%%%%%%%%%%%%%%%%%%%%%%%%%%%%%%%%%%%%%%%%%%%%%%%%%%%%%%%%%%%%%
%% Change options here if you want:
%%
%% How many levels of section head would you like numbered?
%% 0= no section numbers, 1= section, 2= subsection, 3= subsubsection
%%==>>
\setcounter{secnumdepth}{3}

%% How many levels of section head would you like to appear in the
%% Table of Contents?
%% 0= chapter titles, 1= section titles, 2= subsection titles, 
%% 3= subsubsection titles.
%%==>>
\setcounter{tocdepth}{2}

%% Cropmarks? good for final page makeup
%% \docropmarks %% turn cropmarks on

%%%%%%%%%%%%%%%%%%%%%%%%%%%%%%%%%%%%%%%%%%%%%%%%%%%%%%%%%%%%%%%%
%% DRAFT
%
% Uncomment to get double spacing between lines, current date and time
% printed at bottom of page.
% \draft
% (If you want to keep tables from becoming double spaced also uncomment
% this):
% \renewcommand{\arraystretch}{0.6}
%%%%%%%%%%%%%%%%%%%%%%%%%%%%%%

\begin{document}

%%%%%%%%%%%%%%%%%%%%%%%%%%%%%%%%%%%%%%%%%%%%%%%%%%%%%%%%%%%%%%%%
%% Title Pages
%%
%% Wiley will provide title and copyright page, but you can make
%% your own titlepages if you'd like anyway

%% Setting up title pages, type in the appropriate names here:
\booktitle{}
\subtitle{}

\author{}
%or
\authors{}

%% \\ will start a new line.
%% You may add \affil{} for affiliation, ie,
%\authors{Robert M. Groves\\
%\affil{Universitat de les Illes Balears}
%Floyd J. Fowler, Jr.\\
%\affil{University of New Mexico}
%}

%% Print Half Title and Title Page:
\halftitlepage
\titlepage


%%%%%%%%%%%%%%%%%%%%%%%%%%%%%%%%%%%%%%%%%%%%%%%%%%%%%%%%%%%%%%%%
%% Off Print Info

%% Add your info here:
\offprintinfo{title, edition}{author}

%% Can use \\ if title, and edition are too wide, ie,
%% \offprintinfo{Survey Methodology,\\ Second Edition}{Robert M. Groves}


%%%%%%%%%%%%%%%%%%%%%%%%%%%%%%%%%%%%%%%%%%%%%%%%%%%%%%%%%%%%%%%%
%% Copyright Page

\begin{copyrightpage}{year}
Title, etc
\end{copyrightpage}

% Note, you must use \ to start indented lines, ie,
% 
% \begin{copyrightpage}{2004}
% Survey Methodology / Robert M. Groves . . . [et al.].
% \       p. cm.---(Wiley series in survey methodology)
% \    ``Wiley-Interscience."
% \    Includes bibliographical references and index.
% \    ISBN 0-471-48348-6 (pbk.)
% \    1. Surveys---Methodology.  2. Social 
% \  sciences---Research---Statistical methods.  I. Groves, Robert M.  II. %
% Series.\\

% HA31.2.S873 2004
% 001.4'33---dc22                                             2004044064
% \end{copyrightpage}

%%%%%%%%%%%%%%%%%%%%%%%%%%%%%%%%%%%%%%%%%%%%%%%%%%%%%%%%%%%%%%%%
%% Frontmatter >>>>>>>>>>>>>>>>

%%%%%%%%%%%%%%%%%%%%%%%%%%%%%%%%%%%%%%%%%%%%%%%%%%%%%%%%%%%%%%%%
%% Only Dedication (optional) 
%% or Contributor Page for edited books
%% before \tableofcontents

\dedication{}

% ie,
%\dedication{To my parents}

%%%%%%%%%%%%%%%%%%%%%%%%%%%%%%%%%%%%%%%%%%%%%%%%%%%%%%%%%%%%%%%%
%  Contributors Page for Edited Book
%%%%%%%%%%%%%%%%%%%%%%%%%%%%%%%%%%%%%%%%%%%%%%%%%%%%%%%%%%%%%%%%

% If your book has chapters written by different authors,
% you'll need a Contributors page.

% Use \begin{contributors}...\end{contributors} and
% then enter each author with the \name{} command, followed
% by the affiliation information.

% \begin{contributors}
% \name{Masayki Abe,} Fujitsu Laboratories Ltd., Fujitsu Limited, Atsugi,
% Japan

% \name{L. A. Akers,} Center for Solid State Electronics Research, Arizona
% State University, Tempe, Arizona

% \name{G. H. Bernstein,} Department of Electrical and
% Computer Engineering, University of Notre Dame, Notre Dame, South Bend, 
% Indiana; formerly of
% Center for Solid State Electronics Research, Arizona
% State University, Tempe, Arizona 
% \end{contributors}

%%%%%%%%%%%%%%%%%%%%%%%%%%%%%%%%%%%%%%%%%%%%%%%%%%%%%%%%%%%%%%%%
\contentsinbrief %optional
\tableofcontents
% \listoffigures %optional
% \listoftables  %optional

%%%%%%%%%%%%%%%%%%%%%%%%%%%%%%%%%%%%%%%%%%%%%%%%%%%%%%%%%%%%%%%%
% Optional Foreword:

%\begin{foreword}
%text
%\end{foreword}

%%%%%%%%%%%%%%%%%%%%%%%%%%%%%%%%%%%%%%%%%%%%%%%%%%%%%%%%%%%%%%%%
% Optional Preface:

%\begin{preface}
% text
%\prefaceauthor{}
%\where{place\\
% date}
%\end{preface}

% ie,
% \begin{preface}
% This is an example preface.
% \prefaceauthor{R. K. Watts}
% \where{Durham, North Carolina\\
% September, 2004}

%%%%%%%%%%%%%%%%%%%%%%%%%%%%%%%%%%%%%%%%%%%%%%%%%%%%%%%%%%%%%%%%
% Optional Acknowledgments:

% \acknowledgments
% acknowledgment text
% \authorinitials{} % ie, I. R. S.


%%%%%%%%%%%%%%%%%%%%%%%%%%%%%%%%
%% Glossary Type of Environment:

% \begin{glossary}
% \term{<term>}{<description>}
% \end{glossary}

%%%%%%%%%%%%%%%%%%%%%%%%%%%%%%%%
% \begin{acronyms} 
% \acro{<term>}{<description>}
% \end{acronyms}

%%%%%%%%%%%%%%%%%%%%%%%%%%%%%%%%
%% In symbols environment <term> is expected to be in math mode; 
%% if not in math mode, use \term{\hbox{<term>}}

% \begin{symbols}
% \term{<math term>}{<description>}
% \term{\hbox{<non math term>}}Box used when not using a math symbol.
% \end{symbols}

%%%%%%%%%%%%%%%%%%%%%%%%%%%%%%%%
% \begin{introduction}
%\introauthor{<name>}{<affil>}
% Introduction text...
% \end{introduction}

%%%%%%%%%%%%%%%%%%%%%%%%%%%%%%%%%%%%%%%%%%%%%%%%%%%%%%%%%%%%%%%%
%% End for Front Matter, Beginning of text of book  >>>>>>>>>>>

%% Short version of title without \\ may be written in sq. brackets:

%% Optional Part :
\part[Submicron Semiconductor Manufacture]
{Submicron Semiconductor\\ Manufacture}

\chapter[The Submicrometer Silicon MOSFET]
{The Submicrometer\\ Silicon MOSFET}

%%%%%%%%%%%%%%%%%%%%%%%%%%%%%%%%%%%%%%%%%%%%%%%%%%%%%%
%% optional prologue or prologues
% \chapter{Chapter Title}
% \prologue{<text>}{<author attribution>}

%%%%%%%%%%%%%%%%%%%%%%%%%%%%%%%%%%%%%%%%%%%%%
% Edited Book: Author and Affiliation
%%%%%%%%%%%%%%%%%%%%%%%%%%%%%%%%%%%%%%%%%%%%%

% After \chapter{Chapter Title}, you can
% enter the author name and embed the affiliation with
% \chapterauthors{(author name, or names)
% \chapteraffil{(affiliation or affiliations)}
% }    

% For instance:
% \chapter{Chapter Title}
% \chapterauthors{G. Alvarez and R. K. Watts
% \chapteraffil{Carnegie Mellon University, Pittsburgh, Pennsylvania}

% For separate affiliations you can use \affilmark{(number)} after
% the name of a particular author and before the matching affiliation:

% For instance:
% \chapter{Chapter Title}
% \chapterauthors{George Smeal, Ph.D.\affilmark{1}, Sally Smith,
% M.D.\affilmark{2}, and Stanley Kubrick\affilmark{1}
% \chapteraffil{\affilmark{1}AT\&T Bell Laboratories
% Murray Hill, New Jersey\\
% \affilmark{2}Harvard Medical School,
% Boston, Massachusetts}
% }

%%%%%%%%%%%%%%%%%%%%%%%

%% short version of section head, or one without \\ supplied in sq. brackets.

% \section[Introduction and fugue]{Introduction\\ and fugue}
% \subsection[This is the subsection]{This is the\\ subsection}
% \subsubsection{This is the subsubsection}
% \paragraph{This is the paragraph}

% \begin{chapreferences}{widest label}
% \bibitem{<label>}Reference
% \end{chapreferences}

% optional chapter bibliography using BibTeX,
% must also have \usepackage{chapterbib} before \begin{document}
% Must use root file with % vim:ts=4:sw=4
% Copyright (c) 2014 Casper Ti. Vector
% Public domain.

\chapter{�½�}

����һ�¡�

, % vim:ts=4:sw=4
%
% Documentation for pkuthss.
%
% Copyright (c) 2008-2009 solvethis
% Copyright (c) 2010-2014 Casper Ti. Vector
%
% This work may be distributed and/or modified under the conditions of the
% LaTeX Project Public License, either version 1.3 of this license or (at
% your option) any later version.
% The latest version of this license is in
%   http://www.latex-project.org/lppl.txt
% and version 1.3 or later is part of all distributions of LaTeX version
% 2005/12/01 or later.
%
% This work has the LPPL maintenance status `maintained'.
% The current maintainer of this work is Casper Ti. Vector.
%
% This work consists of the following files:
%   pkuthss.tex
%   chap/copyright.tex
%   chap/abstract.tex
%   chap/introduction.tex
%   chap/chap1.tex
%   chap/chap2.tex
%   chap/chap3.tex
%   chap/conclusion.tex
%   chap/encl1.tex
%   chap/acknowledge.tex

\chapter{pkuthss 文档模版提供的功能}
	\section{pkuthss 文档模版提供的文档类和宏包选项}
		\subsection{pkuthss 文档类提供的选项}\label{ssec:options}

		\begin{itemize}
			\item \textbf{\texttt{[no]extra}}:
				用于确定是否自动载入 pkuthss-extra 宏包。
				在默认情况下,pkuthss 文档类将使用 \verb|extra| 选项。
				用户如果不需要自动载入 pkuthss-extra 宏包,
				则需要在载入 pkuthss 时加上 \verb|noextra| 选项。

			\item \textbf{\texttt{[no]uppermark}}:
				是否在页眉中将章节名中的小写字母转换为大写字母。
				就目前而言,
				这样的转换存在着一些较为严重的缺陷\footnote{%
					准确地说是 \texttt{\string\MakeUppercase} 宏的问题,
					其在某些地方的转换不够健壮,
					例如 \texttt{\string\cite\string{ctex\string}}
					会被转换成 \texttt{\string\cite\string{CTEX\string}}。%
				},
				因此不建议使用。
				基于上述考虑,%
				\myemph{%
					pkuthss 文档类默认启用 \texttt{nouppermark} 选项,
					即在不在页眉中使用大写的章节名%
				}。

			\item \textbf{pkuthss-extra 宏包提供的选项}:
				这些选项将被传递给 pkuthss-extra 宏包
				(用户需要启用 \verb|extra| 选项)。
				具体说明参见第 \ref{ssec:extra} 小节。

			\item \textbf{其余文档类选项}:%
				pkuthss 文档类以 ctexbook 文档类为基础,
				其接受的其余所有文档类选项均被传递给 ctexbook。
				其中可能最常用的选项是 \verb|GBK| 和 \verb|UTF8|:
				它们选择源代码使用的字符编码,默认使用 \verb|GBK|。
		\end{itemize}

		例如,如果需要使用 UTF-8 编码撰写论文,
		则需要在导入 pkuthss 文档类时加上 \verb|UTF8| 选项:
\begin{Verbatim}[frame = single]
\documentclass[UTF8, ...]{pkuthss} % “...”代表其它的选项。
\end{Verbatim}

		又例如,文档默认情况下是双面模式,每章都从右页(奇数页)开始。
		如果希望改成一章可以从任意页开始,可以这样设置:
\begin{Verbatim}[frame = single]
\documentclass[openany, ...]{pkuthss} % 每章从任意页开始。
\end{Verbatim}
		但这样设置时左右(奇偶)页的页眉页脚设置仍然是不同的。
		如果需要使左右页的页眉页脚设置一致,可以直接采用单面模式:
\begin{Verbatim}[frame = single]
% 使用 oneside 选项时不需要再指定 openany 选项。
\documentclass[oneside, ...]{pkuthss}
\end{Verbatim}

		\subsection{pkuthss-extra 宏包提供的选项}\label{ssec:extra}

		除非特别说明,
		下面提到的选项中都是不带“\verb|no|”的版本被启用。

		\begin{itemize}
			\item \textbf{\texttt{[no]spacing}}:
				是否采用一些常用的对空白进行调整的版式设定。
				具体地说,启用 \verb|space| 选项后会进行以下几项设置:
			\begin{itemize}
				\item 自动忽略 CJK 文字之间的空白而%
					保留(CJK 文字与英文之间等的)其它空白。
				\item 调用 setspace 宏包以使某些细节处的空间安排更美观。
				\item 设置页芯居中。
				\item 设定行距为 1.41\footnote{%
					为什么是 1.41?因为 $\sqrt{2}\approx1.41$。%
				}。
				\item 使脚注编号和脚注文本之间默认间隔一个空格。
			\end{itemize}

			\item \textbf{\texttt{[no]tightlist}}:
				是否采用比 \LaTeX{} 默认设定更加紧密的枚举环境。
				在枚举环境(itemize、enumerate 和 description)中,
				每个条目的内容较少时,条目往往显得稀疏;
				在参考文献列表中也有类似的现象。
				启用 \verb|tightlist| 选项后,
				将去掉这些环境中额外增加的(垂直)间隔。

			\item \textbf{\texttt{[no]caption}}:
				是否使图表标题使用和正文不同的字体
				(此处设为中文楷书、英文斜体,
				如图 \ref{fig:example} 所示)。
				根据排版中常见的审美原则,
				一般应使图表标题的字体、字号轻
				(例如楷书之于宋体、五号字之于小四号字)于正文,
				图表内容的字体、字号轻于图表标题。

			\begin{figure}[htbp!]
				\centering
				\includegraphics[width = 0.5\textwidth]{pkuword}
				\caption{示例插图}\label{fig:example}
			\end{figure}

			\item \textbf{\texttt{[no]pdftoc}}\footnote{%
					此选项部分等价于 1.4 alpha2 及以前版本 pkuthss-extra 宏包%
					的 \texttt{[no]tocbibind} 选项。
					因为 tocbibind 宏包和 biblatex 宏包冲突,
					pkuthss-extra 宏包不再调用 tocbibind 宏包。%
				}:
				启用 \verb|pdftoc| 选项后,
				用 \verb|\tableofcontents| 命令生成目录时%
				会自动添加“目录”的 pdf 书签。

			\item \textbf{\texttt{[no]spechap}}\footnote{%
					“spechap”是“\textbf{spec}ial \textbf{chap}ter”的缩写。%
				}:
				是否启用第 \ref{ssec:misc} 小节中介绍的 %
				\verb|\specialchap| 命令。

			\item \textbf{\texttt{[no]pdfprop}}:
				是否自动根据设定的论文文档信息(如作者、标题等)
				设置生成的 pdf 文档的相应属性。%
				\myemph{%
					注意:
					该选项实际上是在 \texttt{\string\maketitle} 时生效的,
					这是因为考虑到%
					通常用户在调用 \texttt{\string\maketitle} 前%
					已经设置好所有的文档信息。
					若用户不调用 \texttt{\string\maketitle},
					则需在设定完文档信息之后自行调用%
					第 \ref{ssec:misc} 小节中介绍的 %
					\texttt{\string\setpdfproperties} 命令以完成
					pdf 文档属性的设定。%
				}

			\item \textbf{\texttt{[no]colorlinks}}\footnote{%
					此选项等价于 1.3 及以前版本 pkuthss-extra 宏包%
					的 \texttt{[no]linkcolor} 选项,
					但后来发现这会和 hyperref 宏包的一个同名选项冲突,
					故改为 \texttt{[no]colorlinks}。%
				}:
				是否在生成的 pdf 文档中使用彩色的链接。
		\end{itemize}

		例如,在提交打印版的论文时,
		彩色的链接文字在黑白打印出来之后可能颜色会很浅;
		北大图书馆也要求提交的电子版论文目录使用黑色字体。
		此时用户\myemph{%
			可以启用 pkuthss-extra 宏包的 \texttt{nocolorlinks} 选项,
			使所有的链接变为黑色,以免影响打印或提交%
		}:
\begin{Verbatim}[frame = single]
\documentclass[..., nocolorlinks]{pkuthss} % “...”代表其它的选项。
\end{Verbatim}

	\section{pkuthss 文档模版提供的命令和环境}
		\subsection{设定文档信息的命令}

		这一类命令的语法为
\begin{Verbatim}[frame = single]
\commandname{具体信息} % commandname 为具体命令的名称。
\end{Verbatim}

		这些命令总结如下:
		\begin{itemize}
			\item \texttt{\bfseries\string\ctitle}:设定论文中文标题;
			\item \texttt{\bfseries\string\etitle}:设定论文英文标题;
			\item \texttt{\bfseries\string\cauthor}:设定作者的中文名;
			\item \texttt{\bfseries\string\eauthor}:设定作者的英文名;
			\item \texttt{\bfseries\string\studentid}:设定作者的学号;
			\item \texttt{\bfseries\string\date}:设定日期;
			\item \texttt{\bfseries\string\school}:设定作者的学院名;
			\item \texttt{\bfseries\string\cmajor}:设定作者专业的中文名;
			\item \texttt{\bfseries\string\emajor}:设定作者专业的英文名;
			\item \texttt{\bfseries\string\direction}:设定作者的研究方向;
			\item \texttt{\bfseries\string\cmentor}:设定导师的中文名;
			\item \texttt{\bfseries\string\ementor}:设定导师的英文名;
			\item \texttt{\bfseries\string\ckeywords}:设定中文关键词;
			\item \texttt{\bfseries\string\ekeywords}:设定英文关键词。
		\end{itemize}

		例如,如果要设定专业为“化学”(“Chemistry”),则可以使用以下命令:
\begin{Verbatim}[frame = single]
\cmajor{化学}
\emajor{Chemistry}
\end{Verbatim}

		\subsection{自身存储文档信息的命令}

		这一类命令的语法为
\begin{Verbatim}[frame = single]
% commandname 为具体的命令名。
\renewcommand{\commandname}{具体信息}
\end{Verbatim}

		这些命令总结如下:
		\begin{itemize}
			\item \texttt{\bfseries\string\cuniversity}:大学的中文名。
			\item \texttt{\bfseries\string\euniversity}:大学的英文名。
			\item \texttt{\bfseries\string\cthesisname}:论文类别的中文名。
			\item \texttt{\bfseries\string\ethesisname}:论文类别的英文名。
			\item \texttt{\bfseries\string\cabstractname}:摘要的中文标题。
			\item \texttt{\bfseries\string\eabstractname}:摘要的英文标题。
		\end{itemize}

		例如,
		如果要设定论文的类别为“本科生毕业论文”(“Undergraduate Thesis”),
		则可以使用以下命令:
\begin{Verbatim}[frame = single]
\renewcommand{\cthesisname}{本科生毕业论文}
\renewcommand{\ethesisname}{Undergraduate Thesis}
\end{Verbatim}

		\subsection{以“key = value”格式设置文档信息}

		用户可以通过 \verb|\pkuthssinfo| 命令集中设定文档信息,
		其语法为:
\begin{Verbatim}[frame = single]
% key1、key2、value1、value2 等为具体文档信息的项目名和内容。
\pkuthssinfo{key1 = value1, key2 = value2, ...}
\end{Verbatim}
		其中文档信息的项目名为前面提到的设定文档信息的命令名%
		或自身存储文档信息的命令名(不带反斜杠)。

		当文档信息的内容包含了逗号等有干扰的字符时,
		可以用大括号将这一项文档信息的全部内容括起来。%
		\myemph{%
			我们推荐用户总用大括号将文档信息的内容括起来,
			以避免很多不必要的麻烦。%
		}

		例如,前面提到的文档信息的设置可以集中地写成:
\begin{Verbatim}[frame = single, tabsize = 4]
\pkuthssinfo{
	..., % “...”代表其它的设定。
	cthesisname = {本科生毕业论文},
	ethesisname = {Undergraduate Thesis},
	cmajor = {化学}, emajor = {Chemistry}
}
\end{Verbatim}

		\subsection{pkuthss 文档模版提供的其它命令和环境}\label{ssec:misc}

		\texttt{\bfseries cabstract} 和 \texttt{\bfseries eabstract} %
		环境用于编写中英文摘要。
		用户只需要写摘要的正文;标题、作者、导师、专业等部分会自动生成。

		\texttt{\bfseries\string\specialchap} 命令%
		用于开始不进行标号但计入目录的一章,
		并合理安排其页眉。%
		\myemph{%
			注意:
			需要启用 pkuthss-extra 宏包的 \texttt{spechap} 选项%
			才能使用此命令。
			另外,在此章内的节或小节等命令应使用带星号的版本,
			例如 \texttt{\string\section\string*} 等,
			以免造成章节编号混乱。%
		}%
		例如,本文档中的“序言”一章就是用 \verb|\specialchap{序言}| %
		这条命令开始的。%

		\texttt{\bfseries\string\setpdfproperties} 命令%
		用于根据用户设定的文档信息自动设定生成的 pdf 文档的属性。
		此命令会在用户调用 \verb|\maketitle| 命令时被自动调用,
		因此通常不需要用户自己使用;
		但用户有时可能不需要输出标题页,
		从而不会调用 \verb|\maketitle| 命令,
		此时就需要在设定完文档信息之后调用 \verb|\setpdfproperties|。
		\myemph{%
			注意:
			需要启用 pkuthss-extra 宏包的 \texttt{pdfprop} 选项%
			才能使用此命令。%
		}

		\subsection{从其它文档类和宏包继承的功能}\label{ssec:thirdparty}

		pkuthss 文档类建立在 ctexbook\supercite{ctex} 文档类的基础上,
		并调用了 CJKfntef、%
		graphicx\supercite{graphicx}、geometry\supercite{geometry} 和 %
		fancyhdr\supercite{fancyhdr} 等几个宏包。
		因此,ctexbook 文档类和这些宏包所提供的功能均可以使用。

		例如,用户如果想将目录的标题改为“目{\quad\quad}录”,
		则可以使用 ctexbook 文档类提供的 \verb|\CTEXoptions| 命令:
\begin{Verbatim}[frame = single]
\CTEXoptions[contentsname = {目{\quad\quad}录}]
\end{Verbatim}

		在默认的配置下,%
		pkuthss 文档模版使用作者编写的 %
		biblatex\supercite{biblatex} 样式\supercite{biblatex-caspervector}%
		进行参考文献和引用的排版,
		用户可以使用它以及 biblatex 本身所提供的功能。
		例如,
		用户可以分别使用 \verb|\cite|、\verb|\parencite| 和 \verb|\supercite| %
		生成未格式化的、带方括号的和上标且带方括号的引用标记:
\begin{Verbatim}[frame = single]
\cite{ctex},\parencite{ctex},\supercite{ctex}
\end{Verbatim}
		在本文中将产生“\cite{ctex},\parencite{ctex},\supercite{ctex}”。

		通过更复杂的设置,还可以满足例如被引用的文献按照引用顺序排序,
		而未引用的文献按照英文文献在前、中文文献在后排序这样的需求,
		详见 biblatex-caspervector 的文档\supercite{biblatex-caspervector}。

		pkuthss-extra 宏包可能调用以下这些宏包:
		\begin{itemize}
			\item 启用 \verb|spacing| 选项时会调用 %
				setspace 和 footmisc\supercite{footmisc} 宏包。
			\item 启用 \verb|tightlist| 选项时会调用 %
				enumitem\supercite{enumitem} 宏包。
			\item 启用 \verb|caption| 选项时会调用 %
				caption\supercite{caption} 宏包。
		\end{itemize}
		因此在启用相应选项时,用户可以使用对应宏包所提供的功能。

		\subsection{不建议更改的设置}
		\myemph{%
			pkuthss 文档类中有一些一旦改动就有可能破坏预设排版规划的设置,
			因此不建议更改这些设置,它们是:
			\begin{itemize}
				\item 纸张类型:A4;
				\item 页芯尺寸:%
					$240\,\mathrm{mm}\times150\,\mathrm{mm}$,
					包含页眉、页脚;
				\item 默认字号:小四号。
			\end{itemize}%
		}

	\section{高级设置}\label{sec:advanced}

	pkuthss 文档模版的实现是简洁、清晰、灵活的。
	当一些细节的自定义无法通过模版提供的外部接口实现时,
	我们鼓励用户(在适当理解相关部分代码的前提下)通过修改模版进行自定义。

	一个常见的需求是封面中部分内容(特别是论文的标题、专业和研究方向)太长,
	超出了在预设的空间。
	此时,
	用户可以修改 \verb|pkuthss.cls| 里 \verb|\maketitle| 定义中
	\verb|\pkuthss@int@fillinblank| 宏的参数来改变
	带下划线的空白的行数和行宽,其语法为:
\begin{Verbatim}[frame = single]
\pkuthss@int@fillinblank{行数}{行宽}{内容}
\end{Verbatim}
	例如,如果“研究方向”一栏需要两行的空白,
	可以将 \verb|pkuthss.cls| 里的
\begin{Verbatim}[frame = single]
\pkuthss@int@fillinblank{1}{\pkuthss@tmp@len}{\kaishu\@direction}
\end{Verbatim}
	改为
\begin{Verbatim}[frame = single]
\pkuthss@int@fillinblank{2}{\pkuthss@tmp@len}{\kaishu\@direction}
\end{Verbatim}
	当然,为了美观,可以将多于一行的部分移到封面中作者信息部分的最下方。

 form.
%\bibliographystyle{plain}
%\bibliography{<your .bib file name>}

% optional appendix at the end of a chapter:
% \chapappendix{<chap appendix title>}
% \chapappendix{} % no title

%%%%%%%%%%%%%%%%%%%%%%%%%%%%%%%%%%%%%%%%%%%%%%%%%%%%%%%%%%%%%%%%
%% End Matter >>>>>>>>>>>>>>>>>>

% \appendix{<optional title for appendix at end of book>}
% \appendix{} % appendix without title

% \begin{references}{<widest label>}
% \bibitem{sampref}Here is reference.
% \end{references}

%%%%%%%%%%%%%%%%%%%%%%%%%%%%%%%%%%%%%%%%%%%%%%%%%%%%%%%%%%%%%%%%
%% Optional Problem Sets: Can use this at the end of each chapter or at end
%% of book

% \begin{problems}
% \prob
% text

% \prob
% text

% \subprob
% text

% \subprob
% text

% \prob
% text
% \end{problems}

%%%%%%%%%%%%%%%%%%%%%%%%%%%%%%%%%%%%%%%%%%%%%%%%%%%%%%%%%%%%%%%%
%% Optional Exercises: Can use this at the end of each chapter or at end
%% of book

% \begin{exercises}
% \exer
% text

% \exer
% text

% \subexer
% text

% \subexer
% text

% \exer
% text
% \end{exercises}


%%%%%%%%%%%%%%%%%%%%%%%%%%%%%%%%%%%%%%%%%%%%%%%%%%%%%%%%%%%%%%%%
%% INDEX: Use only one index command set:

%% 1) The default LaTeX Index
\printindex

%% 2) For Topic index and Author index:

% \usepackage{multind}
% \makeindex{topic}
% \makeindex{authors}
% \begin{document}
% ...
% add index terms to your book, ie,
% \index{topic}{A term to go to the topic index}
% \index{authors}{Put this author in the author index}

%% (these are Wiley commands)
%\multiprintindex{topic}{Topic index}
%\multiprintindex{authors}{Author index}

\end{document}

%%%%%%% Demo of section head containing sample macro:
%% To get a macro to expand correctly in a section head, with upper and
%% lower case math, put the definition and set the box 
%% before \begin{document}, so that when it appears in the 
%% table of contents it will also work:

\newcommand{\VT}[1]{\ensuremath{{V_{T#1}}}}

%% use a box to expand the macro before we put it into the section head:

\newbox\sectsavebox
\setbox\sectsavebox=\hbox{\boldmath\VT{xyz}}

%%%%%%%%%%%%%%%%% End Demo


Other commands, and notes on usage:

-----
Possible section head levels:
\section{Introduction}
\subsection{This is subsection}
\subsubsection{This is subsubsection}
\paragraph{This is the paragraph}

-----
Tables:
 Remember to use \centering for a small table and to start the table
 with \hline, use \hline underneath the column headers and at the end of 
 the table, i.e.,

\begin{table}[h]
\caption{Small Table}
\centering
\begin{tabular}{ccc}
\hline
one&two&three\\
\hline
C&D&E\\
\hline
\end{tabular}
\end{table}

For a table that expands to the width of the page, write

\begin{table}
\begin{tabular*}{\textwidth}{@{\extracolsep{\fill}}lcc}
\hline
....
\end{tabular*}
%% Sample table notes:
\begin{tablenotes}
$^a$Refs.~19 and 20.

$^b\kappa, \lambda>1$.
\end{tablenotes}
\end{table}

-----
Algorithm.
Maintains same fonts as text (as opposed to verbatim which uses fixed
width fonts). Space at beginning of line will be maintained if you
use \ at beginning of line.

\begin{algorithm}
{\bf state\_transition algorithm} $\{$
\        for each neuron $j\in\{0,1,\ldots,M-1\}$
\        $\{$   
\            calculate the weighted sum $S_j$ using Eq. (6);
\            if ($S_j>t_j$)
\                    $\{$turn ON neuron; $Y_1=+1\}$   
\            else if ($S_j<t_j$)
\                    $\{$turn OFF neuron; $Y_1=-1\}$   
\            else
\                    $\{$no change in neuron state; $y_j$ remains %
unchanged;$\}$ .
\        $\}$   
$\}$   
\end{algorithm}

-----
Sample quote:
\begin{quote}
quotation...
\end{quote}

-----
Listing samples

\begin{enumerate}
\item
This is the first item in the numbered list.

\item
This is the second item in the numbered list.
\end{enumerate}

\begin{itemize}
\item
This is the first item in the itemized list.

\item
This is the first item in the itemized list.
This is the first item in the itemized list.
This is the first item in the itemized list.
\end{itemize}

\begin{itemize}
\item[]
This is the first item in the itemized list.

\item[]
This is the first item in the itemized list.
This is the first item in the itemized list.
This is the first item in the itemized list.
\end{itemize}

%% Index commands
Author and Topic Indices, See docs.pdf and w-bksamp.pdf
